\documentclass[a4paper, landscape]{article}
\usepackage[top=1.7cm, bottom=0.1cm, left=0.5cm, right=0.5cm, headheight=1cm, footskip=0.3cm]{geometry}
\usepackage{fancyhdr} % Customizable headers and footers
\usepackage{multicol} % Allows for multiple columns
\usepackage[utf8]{inputenc} % Ensures UTF-8 encoding
\usepackage[ngerman]{babel} % Language support for German
\usepackage{amsmath, amsfonts} % Math packages for symbols and fonts
\usepackage{lmodern} % Enhances font quality
\usepackage{graphicx} % For including images
\usepackage[normalem]{ulem} % Provides underlining capabilities
\usepackage[dvipsnames, table]{xcolor} % Adds color options, including for tables
\usepackage{enumitem} % Customizable lists
\usepackage{mathabx} % Additional math symbols
\usepackage{colortbl} % Color options for tables
\usepackage{mathtools} % Additional tools for mathematical typesetting
\usepackage{wallpaper} % Allows adding wallpaper backgrounds
\usepackage{changepage} % Enables adjustments to margins
\usepackage{tikz} % Graphics for drawing and illustrations
\usepackage{tabularx} % Advanced table formatting
\usepackage[skins]{tcolorbox} % Colored boxes with advanced options
\usepackage{lipsum} % For generating dummy text
\usepackage{bbm} % Font package, especially for blackboard bold characters
\usepackage{multirow} % Combines rows in tables
\usepackage{letltxmacro} % Advanced command management
\usepackage{float} % Control float placement
\usepackage{amssymb} % Additional math symbols
\usepackage{algorithm} % Algorithm environment
\usepackage[noend]{algpseudocode} % Pseudocode environment
\usepackage{leftidx} % Left sub/superscripts
\usepackage{empheq} % Emphasized equations
\usepackage{textcomp} % Additional text symbols
\usepackage{calc} % Arithmetic operations within LaTeX lengths
\usepackage[makeroom]{cancel} % Strike through terms in equations
\usepackage{sidecap} % Places captions to the side of figures
\usepackage{physics} % Additional physics-related commands (e.g., Dirac notation)
\usepackage{makecell} % More advanced cell formatting in tables

\usepackage{mathrsfs}

% Custom header format using fancyhdr package
\pagestyle{fancy}
\fancyhf{} % Clears default header/footer
\lhead{\textbf{\subject} - \semester} % Left header
\rhead{\author} % Right header
\fancyfoot[C]{Page - \thepage} % Center the page number at the bottom

% Custom bullet symbol for itemized lists
\renewcommand\textbullet{\ensuremath{\bullet}}

% Command to create circled numbers for lists
\newcommand*\circled[1]{\tikz[baseline=(char.base)]{
        \node[shape=circle,draw,inner sep=1.2pt] (char) {#1};}}

% Increase row height in tables
\renewcommand{\arraystretch}{1.5}

\newtcbox{\mathbox}[1][]{
    nobeforeafter, 
    colframe=white!30!black, % Frame color
    colback=white!20, % Background color
    boxrule=0.5pt, % Border thickness
    arc=0mm, % Rounded corners
    before skip = 1mm,
    right = 1mm,
    left=1mm,
    top=0.5mm,
    bottom=0.5mm,
    center, % Center the content
    #1
}

% Fill the remaining space with blank space (in multicols environment)
\newcommand{\fillblank}{
    \vfill\null\columnbreak
}

% Set the default font family to sans-serif
\renewcommand{\familydefault}{\sfdefault}

% Custom command for creating a cheatsheet environment
\newcommand{\cheatsheet}[1]{
    \let\underbrace\LaTeXunderbrace % Fixes issue with underbrace in custom environments
    \let\overbrace\LaTeXoverbrace % Fixes issue with overbrace in custom environments
    \begin{document}
    \setlength{\columnseprule}{0.4pt} % Sets the rule width between columns
    \footnotesize
    \begin{multicols*}{\cols} % Starts a multicolumn environment
        \setlength{\abovedisplayskip}{\abovedisplayskip - 4mm} % Reduces space above displayed equations
        \setlength{\belowdisplayskip}{\belowdisplayskip - 3mm} % Reduces space below displayed equations
    #1
    \end{multicols*}
    \end{document}
}

% Adjust the space between paragraphs
\setlength{\parskip}{0.1cm}

% Custom section formatting
\renewcommand{\section}[1]{
    \vskip 1pt
    \par\noindent\textbf{\normalsize#1} % Section title in bold, normal size
    \vskip -8pt
    \par\noindent\smash{\rule[-1.5pt]{2cm}{2pt}}\rule{\linewidth  - 2cm}{0.5pt} % Custom underline
    \par\vspace{-0.5mm}
}

% Custom subsection formatting
\renewcommand{\subsection}[2][]{
    \vskip 1pt
    \par\noindent\textbf{\small#2} \textbf{\small#1} % Subsection title in bold, small size
    \vskip -8pt
    \par\noindent\smash{\rule[-1.5pt]{1cm}{2pt}}\rule{\linewidth - 1cm}{0.5pt} % Custom underline
    \par\vspace{-1mm}
}

% Custom subsubsection formatting
\renewcommand{\subsubsection}[1]{
    \vskip 1pt
    \par\noindent\textbf{\footnotesize#1} % Subsubsection title in bold, footnote size
    \vskip -8pt
    \rule{\linewidth}{0.5pt} % Horizontal rule across the width
    \par\vspace{-1mm}
}

% No paragraph indentation
\setlength{\parindent}{0pt}
% Column settings for multicol environment
\setlength\columnsep{2mm}
\setlength{\columnseprule}{0pt}
\setlength{\fboxrule}{0.5pt}
\setlength{\headsep}{0.5em}
\setlength{\footskip}{12.0pt}



\newcommand{\sgn}{\operatorname{sgn}}
\newcommand{\proj}{\operatorname{proj}}
\newcommand{\Span}{\operatorname{Span}}
\newcommand{\Sol}{\operatorname{Sol}}
\newcommand{\Ker}{\operatorname{Ker}}
\newcommand{\Im}{\operatorname{Im}}
\newcommand{\REF}{\operatorname{REF}}
\newcommand{\RREF}{\operatorname{RREF}}


% global setttings
\def\subject{Diskrete Mathematik}
\def\semester{ETH Zürich (252-0025-01L, HS24)}
\def\author{Olivér Kovács}
\def\cols{4}

\cheatsheet{

    \section{Proofs}

    D2.1:
    A \textit{mathematical statement} is a statement that
    is true or false in an absolute, indesputable sense.

    D (Composition of statements):
    \begin{enumerate}
        \item Negation: \(S\) is false.
        \item And: \(S\) and \(T\) are both true.
        \item Or: At least one of \(S\) and \(T\) is true.
        \item Implication: If \(S\) is true, then \(T\) is true.
    \end{enumerate}

    % \subsection{Propositional Logic}

    % \subsection{Predicate Logic}

    % \subsection{Formulas vs Statements}

    % TODO

    \subsection{Proof Patters}

    Direct proof:
    To prove \(S \implies T\) assume \(S\) and show \(T\).

    Indirect proof:
    To prove \(S \implies T\) assume "\(T\) is false" and show
    "\(S\) is false".

    Modus ponens:
    To prove \(S\):
    \begin{enumerate}
        \item Find suitable \(R\).
        \item Prove \(R\).
        \item Prove \(R \implies S\).
    \end{enumerate}

    Case distinction:
    To prove \(S\):
    \begin{enumerate}
        \item Find finite list \(R_1, \ldots, R_k\) of statements.
        \item Prove that at least one of the \(R_i\) is true.
        \item Prove \(R_i \implies S\) for \(i = 1, \ldots, k\).
    \end{enumerate}

    Proof by contradiction:
    To prove \(S\):
    \begin{enumerate}
        \item Find suitable \(T\). 
        \item Prove "\(T\) is false".
        \item Assume "\(S\) is false" and prove \(T\).
    \end{enumerate}

    Pigeonhole principle (T2.10): If a set of \(n\) objects is partitioned
    into \(k < n\) sets, at least one of these sets contains at least
    \(\lceil \frac{n}{k} \rceil\) objects.

    Proof by counterexample.
    
    Proof by (strong) induction.

    \section{Sets, Relations, Functions}

    \subsection{Introduction}

    D3.1:
    The number of elements in a finite set \(A\) is its cardinality \(|A|\).

    D (Russell's paradox): \(R = \{A \mid A \not \in A\}\).

    \subsection{Sets}

    D3.2:
    \(A = B \defiff \forall x (x \in A \lrarr x \in B)\).

    L3.1:
    \(\{a\} = \{b\} \implies a = b\).

    E3.1:
    \((a,b) \defeq \{\{a\}, \{a, b\}\}\).

    % \subsubsection{Subsets}

    D3.3:
    \(A \subseteq B \defiff \forall x(x \in A \to x \in B)\).

    L3.2:
    \(A = B \iff (A \subseteq B) \land (B \subseteq A)\).

    L3.3:
    \(A \subseteq B \land B \subseteq C \implies A \subseteq C\).

    % \subsubsection{Union and Intersection}

    D3.4:
    \(A \stackrel{\cup}{\cap} B \defeq \{x \mid x \in A \stackrel{\lor}{\land} x \in B\}\).
    % \(A \cup B \defeq \{x \mid x \in A \lor x \in B\}\), \\
    % \(A \cap B \defeq \{x \mid x \in A \land x \in B\}\).

    D3.5:
    \(B \setminus A \defeq \{x \in B \mid x \not \in A\}\).

    T3.4:
    For any sets \(A, B, C\) the following laws hold:
    
    \begin{center}
        \begin{tabular}{|l|l|}
            \hline
            name & law \\
            \hline
            \textit{idempot.} & \(A \stackrel{\cup}{\cap} A = A\) \\
            \textit{commut.} & \(A \stackrel{\cup}{\cap} B = B \stackrel{\cup}{\cap} A\) \\
            \textit{assoc.} & \(A \stackrel{\cup}{\cap} (B \stackrel{\cup}{\cap} C) =
                (A \stackrel{\cup}{\cap} B) \stackrel{\cup}{\cap} C\) \\
            \textit{absorp.} & \(A \stackrel{\cup}{\cap} (A \stackrel{\cap}{\cup} B) = A\) \\
            \textit{distrib.} & \(A \stackrel{\cup}{\cap} (B \stackrel{\cap}{\cup} C) =
            (A \stackrel{\cup}{\cap} B) \stackrel{\cap}{\cup} (A \stackrel{\cup}{\cap} C)\) \\
            \textit{consist.} &  \(A \subseteq B \iff\) \\
            & \(A \cap B = A \iff A \cup B = B\) \\
            \hline
        \end{tabular}
    \end{center}

    % \subsubsection{The empty set}
    D3.6:
    The set \(A\) is called \textit{empty} if \(\forall x \lnot (x \in A)\).

    L3.5:
    There is only one empty set, denoted as \(\es\) or \(\{\}\).

    L3.6:
    \(\forall A (\es \subseteq A)\).

    % \subsubsection{Constructing sets from the empty set}

    % \subsubsection{Construction of the natural numbers}

    R (Construction of \(\mathbb N\)): \\
    \(\mathbf 0 \defeq \varnothing\) 
    \(\mathbf 1 \defeq \{ \varnothing \}\)
    \(\mathbf 2 \defeq \{ \varnothing, \{ \varnothing \} \}\)
    \(\mathbf 3 \defeq \{ \varnothing, \{ \varnothing \}, \{ \varnothing, \{ \varnothing \} \} \}\) \\
    The successor of \(\mathbf n\) is defined as
    \(s(\mathbf n) \defeq \mathbf n \cup \{ \mathbf n\}\). \\
    An operation \(\mathbf +\) can be defined recursively as \(\mathbf {m + 0} \defeq \mathbf m\)
    and \(\mathbf {m +} s(\mathbf n) \defeq s(\mathbf {m + n})\).

    % \subsubsection{The power set of a set}

    D3.7:
    \(\mathcal P(A) \defeq \{S \mid S \subseteq A\}\).

    % \subsubsection{The Cartesian product of sets}

    D3.8:
    \(A \times B \defeq \{(a, b) \mid a \in A \land b \in B\}\).

    \subsection{Relations}

    % \subsubsection{Concept}

    D3.9:
    A (binary) \textit{relation} \(\rho\) from \(A\) to \(B\) is a subset of
    \(A \times B\). We also write \(a \rho b\) (\(a \cancel \rho b\)) instead of
    \((a,b) \in \rho\) (\((a,b) \not \in \rho\)).

    D3.10:
    \(\id_A = \{(a, a) \mid a \in A\}\).

    % \subsubsection{Representation}

    Representations of \(\rho\):
    \begin{itemize}
        \item \(|A| \times |B|\) matrix \(M^\rho\) with \(M_{a,b}^\rho = 1\) iff
            \(a \rho b\).
        \item Directed graph with the vertices labeled with elements of \(A\) and \(B\)
            that contains an edge from \(a\) to \(b\) iff \(a \rho b\).
    \end{itemize}

    % \subsubsection{Set operations on Relations}

    % \subsubsection{Inverse of a Relation}

    D3.11:
    \(\hat \rho \defeq \{(b, a) \mid (a, b) \in \rho\}\).

    % \subsubsection{Composition of Relations}

    D3.12:
    \(\rho \circ \sigma \defeq \{(a, c) \mid \exists b
        ((a,b) \in \rho) \land (b,c) \in \sigma)\}\).

    L3.7:
    \(\rho \circ (\sigma \circ \phi) = (\rho \circ \sigma) \circ \phi\).

    L3.8:
    \(\widehat{\rho \sigma} = \hat \sigma \hat \rho\).

    \subsubsection{Special properties of Relations}

    D3.13-D3.17, L3.9:
    
    \begin{center}
        \begin{tabular}{|l|l|l|}
            \hline
            name & condition & set \\
            \hline
            \textit{reflexive} & \(a \rho a\) & \(\id \subseteq \rho\) \\
            \textit{symmetric} & \(a \rho b \iff b \rho a\) & \(\rho = \hat \rho\) \\
            \textit{antisym.} & \(a \rho b \land b \rho a \implies a = b\) & \(\rho \cap \hat \rho \subseteq \id\) \\
            \textit{transitive} & \(a \rho b \land b \rho c \implies a \rho c\) & \(\rho^2 \subseteq \rho\) \\
            \hline
        \end{tabular}
    \end{center}

    \subsubsection{}
    
    % \subsubsection{Transitive closure}
    
    D3.18:
    The \textit{transitive closure} of a relation \(\rho\) on a set \(A\) is
    \(\rho^* = \bigcup_{n \in \mathbb Z^+} \rho^n\).

    \subsection{Equivalence relations}

    D3.19:
    An \textit{equivalence relation} is a relation on a set \(A\) that is
    reflexive, symmetric, and transitive.

    D3.20:
    For an equivalence relation \(\theta\) on \(A\) and \(a \in A\) the
    \textit{equivalence class} of \(a\) is
    \([a]_\theta \defeq \{b \in A \mid b \theta a\}\).

    L3.10:
    The intersection of equivalence relations is an equivalence relation.

    D3.21: A \textit{partition} of a set \(A\) is a set of mutually disjoint
    subsets of \(A\) that cover \(A\).

    D3.22:
    The set of equivalence classes of an equivalence relation \(\theta\)
    denoted by \(A \slash \theta \defeq \{[a]_\theta \mid a \in A\}\)
    is called the \textit{quotient set} of \(A\) by \(\theta\) or
    \(A\) \textit{modulo} \(\theta\) or \(A\) mod \(\theta\).

    T3.11:
    The set \(A \slash \theta\) is a partition of \(A\).
 
    \subsection{Partial order relations}

    D3.23:
    A \textit{partial order} on a set \(A\) is a relation \(\preceq\) that is
    reflexive, antisymmetric, and transitive. Then \((A; \preceq)\) is called
    a \textit{partially ordered set} or \textit{poset}.

    D:
    \(a \prec b \defiff a \preceq b \land a \neq b\).

    D3.24:
    For a poset \((A; \preceq)\), two elements \(a\), \(b\) are called
    \textit{comparable} if \(a \preceq b\) or \(b \preceq a\); otherwise
    \textit{incomparable}.

    D3.25:
    If any two elements of a poset \((A; \preceq)\) are comparable, then
    \(A\) is called \textit{totally ordered} by \(\preceq\).

    D3.26:
    In a poset \((A; \preceq)\) \(b\) is said to \textit{cover} \(a\) if
    \(a \prec b\) and there exists no \(c\) with \(a \prec c\) and \(c \prec b\).

    D3.27:
    The \textit{Hasse diagram} of a (finite) poset \((A; \preceq)\) is the directed
    graph whose vertices are labeled with the elements of \(A\) and where there is an
    edge from \(a\) to \(b\) iff \(b\) covers \(a\).

    D3.28:
    The \textit{direct product} of posets \((A; \preceq)\) and \((B; \sqsubseteq)\)
    denoted \((A; \preceq) \times (B; \sqsubseteq)\) is \(A \times B\) with
    the relation \(\le\) defined by \((a_1, b_1) \le (a_2, b_2) \defiff
    a_1 \preceq a_2 \land b_1 \sqsubseteq b_2\).

    T3.12:
    \((A; \preceq) \times (B; \sqsubseteq)\) is a poset.

    T3.13:
    For the posets \((A; \preceq)\) and \((B; \sqsubseteq)\), the relation
    \(\le_{\lex}\) defined on \(A \times B\) by \((a_1, b_1) \le_{\lex} (a_2, b_2)
    \defiff a_1 \preceq a_2 \lor (a_1 = a_2 \land b_1 \sqsubseteq b_2)\).

    D3.29:
    Let \((A; \preceq)\) be a poset and let \(S \subseteq A\). Then:
    \begin{enumerate}
        \item \(a \in A\) is a \textit{minimal} (\textit{maximal}) element of \(A\)
            iff there exists no \(b \in A\) with \(b \prec a\) (\(b \succ a\)).
        \item \(a \in A\) is the \textit{least} (\textit{greatest}) element of \(A\)
            iff \(a \preceq b\) (\(a \succeq b\)) for all \(b \in A\).
        \item \(a \in A\) is a \textit{lower} (\textit{upper}) \textit{bound}
            of \(S\) iff \(a \preceq b\) (\(a \succeq b\)) for all \(b \in S\).
        \item \(a \in A\) is a \textit{greatest lower bound} (\textit{least upper bound})
            of \(S\) iff \(a\) is the greatest (least) element of the set of all
            lower (upper) bounds of \(S\).
    \end{enumerate}

    D3.30:
    A poset \((A; \preceq)\) is \textit{well-ordered} if it is totally ordered and
    if every non-empty subset of \(A\) has a least element.

    D3.31:
    Let \((A; \preceq)\) be a poset. If \(a\) and \(b\) have a greatest lower bound,
    then it is called the \textit{meet} of \(a\) and \(b\), often denoted
    \(a \land b\). If \(a\) and \(b\) have a least upper bound, then it is called the
    \textit{join} of \(a\) and \(b\), often denoted \(a \lor b\).

    D3.32:
    A poset \((A; \preceq)\) in which every pair of elements has a meet and join
    is called a \textit{lattice}.

    \subsection{Functions}

    D3.33:
    A \textit{function} \(f \colon A \to B\) from a \textit{domain} \(A\) to a
    \textit{codomain} \(B\) is a relation from \(A\) to \(B\) with the special
    properties:
    \begin{enumerate}
        \item \(\forall a \in A \exists b \in B a f b\),
        \item \(\forall a \in A \forall b, b' \in B (a f b \land a f b' \to b = b')\).
    \end{enumerate}
    (\(f\) is totally defined and well-defined).

    D3.34:
    The set of all functions \(A \to B\) is denoted as \(B^A\).

    D3.35:
    A \textit{partial function} \(A \to B\) is a relation from \(A\) to \(B\)
    such that condition 2. in [D3.33] holds.

    D3.36:
    For \(f \colon A \to B\) and \(S \subseteq A\), the \textit{image} of \(S\)
    under \(f\) is \(f(S) \defeq \{f(s) \mid s \in S\}\).

    D3.37:
    The subset \(f(A)\) of \(B\) is called the \textit{image} of \(f\).

    D3.38:
    For \(T \subseteq B\), the \textit{preimage} of \(T\) is
    \(f^{-1}(T) \defeq \{a \in A \mid f(a) \in T\}\).

    D3.39:
    A function \(f \colon A \to B\) is called:
    \begin{enumerate}
        \item \textit{injective} if \(a \neq a' \implies f(a) \neq f(a')\),
        \item \textit{surjective} if \(f(A) = B\).
        \item \textit{bijective} if it is both injective and surjective.
    \end{enumerate}

    D3.40:
    For a bijective function \(f\) the \textit{inverse} is called the
    inverse function of \(f\), denoted \(f^{-1}\).

    D3.41:
    The \textit{composition} of functions \(f \colon A \to B\),
    \(g \colon B \to C\), denoted \(g \circ f\), is defined by
    \((g \circ f)(a) = g(f(a))\).

    L3.14:
    \((h \circ g) \circ f = h \circ (g \circ f)\).

    \subsection{Countability}

    D3.42:
    Let \(A, B\) be sets.
    \begin{enumerate}
        \item \(A, B\) are \textit{equinumerous}, denoted \(A \sim B\), iff
            there exists a bijection \(A \to B\).
        \item \(B\) \textit{dominates} \(A\), denoted \(A \preceq B\),
            if \(A \sim C\) for some \(C \subseteq B\), or equivalently,
            there exits an injection \(A \to B\).
        \item \(A\) is called \textit{countable} iff \(A \preceq \mathbb N\)
            and \textit{uncountable} otherwise.
    \end{enumerate}

    L3.15:
    \begin{enumerate}
        \item The relation \(\sim\) is an equivalence relation.
        \item The relation \(\preceq\) is transitive.
        \item \(A \subseteq B \implies A \preceq B\).
    \end{enumerate}

    T3.16:
    \(A \preceq B \land B \preceq A \implies A \sim B\).

    T3.17:
    A set \(A\) is countable iff it is finite or if \(A \sim \mathbb N\).

    T3.18:
    The set \(\{0, 1\}^* \defeq \{\epsilon, 0, 1, 00, 01, 10, \ldots\}\)
    of finite sequences in countable.

    T3.19:
    The set \(\mathbb N \times \mathbb N\) is countable.

    % C3.20:
    % The Cartesian product of countable sets is countable.

    % C3.21:
    % \(\mathbb Q\) is countable.

    T3.22:
    Let \(A\) be a countable set.
    \begin{enumerate}
        \item For any \(n \in \mathbb N\), the set \(A^n\) is countable.
        \item The union of a countable list of of countable
            sets is countable.
        \item The set \(A^*\) is countable.
    \end{enumerate}

    D3.43:
    Let \(\{0, 1\}^\infty\) denote the set of semi-infinte binary sequences,
    or equivalently, of functions \(\mathbb N \to \{0, 1\}\).

    T3.23:
    The set \(\{0, 1\}^\infty\) is uncountable.

    D3.44:
    A function \(f \colon \mathbb N \to \{0, 1\}\) is called
    \textit{computable} iff there is a program that, for every
    \(n \in \mathbb N\), when given \(n\) as input, outputs \(f(n)\).

    C3.24:
    There are uncomputable functions \(\mathbb N \to \{0, 1\}\).

    \section{Number theory}

    \subsection{Divisors and Division}

    D4.1: Divisibility.

    T4.1 (Euclid):
    For all \(a \in \mathbb Z\) and \(d \neq 0\) there exist
    unique \(q, r \in \mathbb Z\) satisfying \(a = dq + r\) and
    \(0 \le r < |d|\).

    D4.2:
    For \(a, b \in \mathbb Z\) (not both 0) \(d\) is called a
    \textit{greatest common divisor} of \(a\) and \(b\) if every common divisor
    of \(a\) and \(b\) divides \(d\), i.e. if
    \(d \mid a \land d \mid b \land \forall c((c \mid a \land c \mid b) \to c \mid d)\).

    D4.3:
    For \(a, b \in \mathbb Z\) (not both 0) one denotes the unique positive greatest
    common divisor by \(\gcd(a, b)\) and calls it \textit{the} greatest common
    divisor. If \(\gcd(a, b) = 1\), then \(a\) and \(b\) are called
    \textit{relatively prime}.

    L4.2:
    For \(m, n, q \in \mathbb Z\) we have \(\gcd(m, n - qm) = \gcd(m, n)\).

    D4.4:
    For \(a, b \in \mathbb Z\), the \textit{ideal generated by \(a\) and \(b\)} is
    \((a, b) \defeq \{ua + vb \mid u, v \in \mathbb Z\}\).

    L4.3:
    For \(a, b \in \mathbb Z\) there exists \(d \in \mathbb Z\) such that
    \((a, b) = (d)\).

    L4.4:
    Let \(a, b \in \mathbb Z\) (not both 0). If \((a, b) = (d)\) then \(d\) is a
    greatest common divisor of \(a\) and \(b\).

    D4.5:
    The \textit{least common multiple} \(l\) of \(a, b \in \mathbb Z^+\),
    denoted \(l = \lcm(a, b)\), is the common multiple of \(a\) and \(b\)
    which divides every common multiple of \(a\) and \(b\), i.e.
    \(a \mid l \land b \mid l \land \forall m((a \mid m \land b \mid m) \to l \mid m)\).

    \subsection{Factorization into primes}

    % \subsubsection{Primes and the fundamental theorem of arithmetic}

    D4.6:
    \(p \in \mathbb Z_{>1}\) is called \textit{prime} iff the only positive
    divisors of \(p\) are 1 and \(p\). An \(x \in \mathbb Z_{>1}\) that is not
    prime is called \textit{composite}.

    T4.6:
    Every positive integer can be written uniquely as the product of primes.

    % \subsubsection{Proof of the fundamental theorem of arithmetic*}

    \subsubsection{Expressing gcd and lcm}

    T:
    Let \(a = \prod_i p_i^{e_i}\), \(b = \prod_i p_i^{f_i}\).
    Then \(\gcd(a, b) = \prod_i p_i^{\min(e_i, f_i)}\) and
    \(\lcm(a, b) = \prod_i p_i^{\max(e_i, f_i)}\).

    % \subsubsection{Non-triviality of unique factorization*}

    % \subsubsection{Irrationality of roots*}

    % \subsubsection{A digression to music theory*}

    % \subsection{Some basic fact about primes*}

    \subsection{Congruences and modular arithmetic}

    % \subsubsection{Modular congruences}

    D4.8:
    Let \(a, b, m \in \mathbb Z\) with \(m \ge 1\).
    \(a \equiv_m b \defiff m \mid (a - b)\).

    L4.13:
    For \(m \ge 1\), \(\equiv_m\) is an equivalence relation on \(\mathbb Z\).

    L4.14:
    If \(a \equiv_m b\) and \(c \equiv_m d\), then
    \(a + c \equiv_m b + d\) and \(ac \equiv_m bd\).

    % C4.15: trivial

    % \subsubsection{Modular arithmetic}

    L4.16:
    For any \(a, b, m \in \mathbb Z\) with \(m \ge 1\),
    \begin{enumerate}
        \item \(a \equiv_m R_m(a)\),
        \item \(a \equiv_m b \iff R_m(a) = R_m(b)\).
    \end{enumerate}

    % C4.17: trivial

    % \subsubsection{Multiplicative inverse}

    L4.18:
    The congruence equation \(ax \equiv_m 1\) has a (unique) solution
    \(x \in \mathbb Z_m\) iff \(\gcd(a, m) = 1\).

    D4.9:
    If \(\gcd(a, m) = 1\), the unique solution \(x \in \mathbb Z_m\)
    to the congruence equation \(ax \equiv_m 1\) is called the
    \textit{multiplicative inverse of} \(a\) \textit{modulo} \(m\).
    One also uses the notation \(x \equiv_m a^{-1}\).

    % \subsubsection{The Chinese remainder theorem}

    T4.19:
    Let \(m_1, \ldots, m_r\) be pairwise relatively prime integers and let
    \(M = \prod_{i=1}^r m_i\). For every list \(a_1, \ldots, a_r\) with
    \(0 \le a_i < m_i\) for \(1 \le i \le r\), the system of congruence equations
    \begin{align*}
        x & \equiv_{m_1} a_1 \\
        & \cdots \\
        x & \equiv_{m_r} a_r \\
    \end{align*}
    for \(x\) has a unique solution \(x\) satisfying \(0 \le x < M\).

    \subsubsection{Application: Diffie-Hellman key exchange}

    D (Diffie-Hellman): Let \(p \in \mathbb P\) and \(g\) be public.

    \begin{center}
        \begin{tabular}{l l l}
            Alice & insecure & Bob \\
            select \(x_A \in [p-2]\) & & select \(x_B \in [p-2]\) \\
            \(y_A := R_p(g^{x_A})\) & & \(y_B := R_p(g^{x_B})\) \\
            & \(\xrightarrow{\quad y_A \quad}\) & \\
            & \(\xleftarrow{\quad y_B \quad}\) & \\
            \(k_{AB} := R_p(y_B^{x_A})\) & & \(k_{BA} := R_p(y_A^{x_B})\) \\
        \end{tabular}
    \end{center}
    \[
        k_{AB} \equiv_p k_{BA}
    \]

    \section{Algebra}

    \subsection{Introduction}

    D5.1:
    An \textit{operation} on a set \(S\) is a function \(S^n \to S\),
    where \(n \ge 0\) is called the \textit{arity} of the operation.

    D5.2:
    An \textit{algebra} is a pair \(\langle S; \Omega \rangle\) where
    \(S\) is a set (\textit{carrier}) and \(\Omega\) is a list of operations
    on \(S\).

    \subsection{Monoids and groups}

    D5.3:
    A \textit{left [right] neutral element} of an algebra
    \(\langle S; * \rangle\) is an element \(e \in S\) such tha
    \(e * a = a\) [\(a * e = a\)] for all \(a \in S\). 

    L5.1:
    If \(\langle S; * \rangle\) has both a left and right neutral element,
    then they are equal.

    D5.4:
    A binary operation \(*\) on a set \(S\) is \textit{associative} if
    \(a * (b * c) = (a * b) * c\) for all \(a, b, c \in S\).

    D5.5:
    A \textit{monoid} is an algebra \(\langle M; *, e \rangle\) where \(*\)
    is associative and \(e\) is the neutral element.

    D5.6:
    A \textit{left [right] inverse element} of an element \(a\) in
    \(\langle S; *, e \rangle\) is an element \(b \in S\) such that
    \(b * a = e\) [\(a * b = e\)].

    L5.2:
    In a monoid \(\langle M; *, e \rangle\),  if \(a \in M\) has a left and right
    inverse, then they are equal.

    D5.7:
    A \textit{group} is an algebra \(\langle G; *, \hat ~, e \rangle\)
    satisfying the following axioms:
    \begin{enumerate}
        \item \(*\) is associative.
        \item \(e\) is a neutral element.
        \item Every \(a \in G\) has an inverse element \(\hat a\).
    \end{enumerate}

    D5.8:
    A group \(\langle G; * \rangle\) (or monoid) is called \textit{commutative}
    if \(a * b = b * a\) for all \(a, b \in G\).

    L5.3:
    For a group \(\langle G; *, \hat ~, e \rangle\), we have for all
    \(a, b, c \in G\):
    \begin{enumerate}
        \item \(\widehat{\hat a} = a\).
        \item \(\widehat{a * b} = \hat b * \hat a\).
        \item Left cancellation: \(a * b = a * c \implies b = c\).
        \item Right cancellation: \(b * a = c * a \implies b = c\).
        \item The equation \(a * x = b\) has a unique soultion \(x\) for
            any \(a, b\).
    \end{enumerate}

    \subsection{The structure of groups}

    D5.9:
    The \textit{direct product} of \(n\) groups \(\langle G_1, *_1 \rangle,
    \ldots, \langle G_n, *_n \rangle\) is the algebra
    \(\langle G_1 \times \cdots \times G_n, \star \rangle\),
    where the operation \(\star\) is component-wise:
    \((a_1, \ldots, a_n) \star (b_1, \ldots, b_n) =
    (a_1 *_1 b_1, \ldots, a_n *_n b_n)\).

    L5.4:
    \(\langle G_1 \times \cdots \times G_n, \star \rangle\) is a group,
    where the neutral element and the inversion operation are
    component-wise in the respective groups.

    D5.10:
    For groups \(\langle G; *, \hat ~, e \rangle\) and
    \(\langle H; \star, \tilde ~, e' \rangle\), a function
    \(\psi \colon G \to H\) is called a \textit{group homomorphism} iff
    for all \(a\) and \(b\) we have \(\psi(a * b) = \psi(a) \star \psi(b)\).
    Iff \(\psi\) is a bijection from \(G\) to \(H\), then it is called an
    \textit{isomorphism}, and we write \(A \simeq H\).

    L5.5:
    A \(\psi\) in [D5.10] satisfies:
    \begin{enumerate}
        \item \(\psi(e) = e'\),
        \item \(\psi(\hat a) = \widetilde{\psi(a)}\) for all \(a\).
    \end{enumerate}

    D5.11:
    \(H \subseteq G\) of \(\langle G; *, \hat ~, e \rangle\) is called a
    \textit{subgoup} of \(G\) iff \(\langle H; *, \hat ~, e \rangle\) is a
    group (closed).

    D5.12:
    Let \(G\) be a group and \(a \in G\). The \textit{order}
    of \(a\), denoted \(\ord(a)\) is the least \(m \ge 1\) such that
    \(a^m = e\) if such an \(m\) exists, and \(\infty\) otherwise.

    D5.13:
    For a finite group \(G\), \(|G|\) is called the \textit{order} of \(G\).

    D5.14:
    For a group \(G\) and \(a \in G\) the \textit{group generated by} \(a\)
    is \(\langle a \rangle \defeq \{a^n \mid n \in \mathbb Z\}\).

    D5.15:
    A group \(G = \langle g \rangle\) is called \textit{cyclic} and \(g\)
    is called a \textit{generator} of \(G\).

    T5.7:
    A cyclic group of order \(n\) is isomorphic to
    \(\langle \mathbb Z_n; \oplus \rangle\).

    T5.8 (Lagrange): Let \(G\) be a finite group and let \(H\) be a subgoup of
    \(G\). Then \(|H|\) divides \(|G|\).

    % C5.9: trivial

    C5.10:
    Let \(G\) be a finite group. Then \(a^{|G|} = e\) for every \(a \in G\).

    C5.11:
    Every group of prime order is cyclic, and every element except the neutral
    element is a generator.

    D5.16:
    \(\mathbb Z_m^* \defeq \{a \in \mathbb Z_m | \gcd(a, m) = 1\}\).

    D5.17 (Euler function):
    \(\varphi(m) = |\mathbb Z_m^*|\).

    L5.12:
    If the prime factorization of \(m\) is \(m = \prod_{i=1}^r p_i^{e_i}\), then
    \(\varphi(m) = \prod_{i=1}^r (p_i - 1) p_i^{e_i - 1}\).

    T5.13:
    \(\langle \mathbb Z_m^*; \odot, ~^{-1}, 1 \rangle\) is a group.

    C5.14 (Fermat, Euler):
    For all \(m \ge 2\) and all \(a\) with \(\gcd(a, m) = 1\) we have
    \(a^{\varphi(m)} \equiv_m 1\).
    In particular, for \(p \in \mathbb P\) and \(p \not \mid a\) we have
    \(a^{p - 1} \equiv_p 1\).

    \subsection{Application: RSA Public-key cryptography}

    T5.16:
    Let \(G\) be some finite group and let \(e \in \mathbb Z\)
    be relatively prime to \(|G|\). The function \(x \mapsto x^e\)
    is a bijection and the unique \(e\)-th root of \(y \in G\),
    namely \(x \in G\) satisfying \(x^e = y\) is \(x = y^d\),
    where \(ed \equiv_{|G|} 1\).

    D (RSA):
    \begin{center}
        \begin{tabular}{l c l}
            Alice & insecure & Bob \\
            generate \(p, q \in \mathbb P\) & & \\
            \(m = pq\) & & \\
            \(\lambda = (p - 1)(q - 1)\) & & \\
            select \(e\) & & \\
            \(d \equiv_\lambda e^{-1}\) & \(\xrightarrow{\quad n, e \quad}\) & \(m \in [n-1]\) \\
            \(m = R_n(c^d)\) & \(\xleftarrow{\quad \ c \ \quad }\) & \(c = R_n(m^e)\) \\
        \end{tabular}
    \end{center}

    \subsection{Rings and fields}

    D5.18:
    A \textit{ring} \(\langle R; +, -, 0, \cdot, 1 \rangle\) is an algebra for which
    \begin{enumerate}
        \item \(\langle R; +, -, 0 \rangle\) is a commutative group.
        \item \(\langle R; \cdot, 1 \rangle\) is a monoid.
        \item \(a(b + c) = (ab) + (ac)\) and \((b + c)a = (ba) + (ca)\)
            for all \(a, b, c \in \mathbb R\).
    \end{enumerate}
    A ring is called \textit{commutative} if multiplication is commutative:
    \(ab = ba\).

    L5.17:
    For any ring \(\langle R; +, -, 0, \cdot, 1 \rangle\), and for all
    \(a, b, \in \mathbb R\),
    \begin{enumerate}
        \item \(0a = a0 = 0\).
        \item \((-a)b = -(ab)\).
        \item \((-a)(-b) = ab\).
        \item If \(R\) is non-trivial, then \(1 \neq 0\).
    \end{enumerate}

    D5.19:
    The \textit{characteristic} of a ring is the order of 1 in the additive
    group if it is finite, and 0 otherwise.

    D5.20:
    An element \(u\) or a ring \(R\) is called a \textit{unit} if \(u\)
    is invertible.
    The set of units of \(R\) is denoted by \(R^*\).

    L5.18:
    For a ring \(R\), \(R^*\) is a group.

    D5.21:
    For \(a, b \in R\) we say that \(a\) \textit{divides} \(b\), denoted
    \(a \mid b\), if there exists \(c \in R\) such that \(b = ac\). In the
    case, \(a\) is called a \textit{divisor} of \(b\) and \(b\) is a
    \textit{multiple} of \(a\).

    L5.19:
    In any commutative ring,
    \begin{enumerate}
        \item If \(a \mid b\) and \(b \mid c\) then \(a \mid c\).
        \item If \(a \mid b\), then \(a \mid bc\).
        \item If \(a \mid b\) and \(a \mid c\), then \(a \mid (b + c)\).
    \end{enumerate}

    D5.22:
    gcd in \(R\).

    D5.23:
    An element \(a \neq 0\) of a commutative ring \(R\) is called a
    \textit{zerodivisor} if \(ab = 0\) for some \(b \neq 0\) in \(R\).

    D5.24:
    An \textit{integral domain} \(D\) is a (nontrivial) commutative ring
    without zerodivisors: For all \(a, b \in D\) we have
    \(ab = 0 \implies a = 0 \lor b = 0\).

    L5.20: In an integral domain, if \(a \mid b\), then \(c\) with \(b = ac\)
    is unique (and is denoted by \(c = \frac{b}{a}\) or \(c = b \slash a\) and
    called quotient).

    D5.25:
    A \textit{polynomial} \(a(x)\) over a commutative ring \(R\) in the
    indeterminate \(x\) is a formal expression of the form
    \(a(x) = a_d x^d + \cdots + a_1 x + a_0\) for some non-negative integer
    \(d\), with \(a_i \in R\). The \textit{degree} of \(a(x)\), denoted
    \(\deg(a(x))\), is the greatest \(i\) for which \(a_i \neq 0\).
    The special polynomial 0 is defined to have degree "minus infinity".
    Let \(R[x]\) denote the set of polynomials (in \(x\)) over \(R\).

    T5.21:
    For any commutative ring \(R\), \(R[x]\) is a commutative ring.

    L5.22:
    Let \(D\) be an integral domain. Then
    \begin{enumerate}
        \item \(D[x]\) is an integral domain.
        \item The degree of the product of two polynomials is the sum of
            their degrees.
        \item The units of \(D[x]\) are the constant polynomials that are
            units of \(D\): \(D[x]^* = D^*\).
    \end{enumerate}

    D5.26:
    A field is a nontrivial commutative ring \(F\) in which every nonzero
    element is a unit, i.e., \(F^* = F \setminus \{0\}\).

    T5.23:
    \(\mathbb Z_p\) is a field iff \(p\) is prime.

    T5.24:
    A field is an integral domain.

    \subsection{Polynomials over a field}

    D5.27:
    A polynomial \(a(x) \in F[x]\) is called \textit{monic} if the leading
    coefficient is 1.

    D5.28:
    A polynomial \(a(x) \in F[x]\) with degree at least 1 is called
    \textit{irreducible} if it is divisible only by constant polynomials
    and by constant multiples of \(a(x)\).

    D5.29:
    The monic polynomial \(g(x)\) of largest degree such that
    \(g(x) \mid a(x)\) and \(g(x) \mid b(x)\) is called \textit{the}
    greatest common divisor of \(a(x)\) and \(b(x)\), denoted
    \(\gcd(a(x), b(x))\).

    T5.25:
    Let \(F\) be a field. For \(a(x)\) and \(b(x) \neq 0\) in \(F[x]\)
    there exist a unique \(q(x)\) (the quotient) and a unique \(r(x)\)
    (the remainder) such that
    \(a(x) = b(x) q(x) + r(x)\) and \(\deg(r(x)) < \deg(b(x))\).

    \subsection{Polynomials as functions}

    % L5.28 trivial

    D5.33:
    Let \(a(x) \in R[x]\). An element \(\alpha \in R\) for which \(a(\alpha) = 0\)
    is called a \textit{root} of \(a(x)\).

    L5.29:
    For a field \(F\), \(\alpha \in F\) is a root of \(a(x)\) iff
    \(x - \alpha\) divides \(a(x)\).

    C5.30:
    A polynomial \(a(x)\) of degree 2 or 3 over a field \(F\) is irreducible
    iff it has no root.

    T5.31:
    For a field \(F\), a nonzero polynomial \(a(x) \in F[x]\) of degree \(d\)
    has at most \(d\) roots.

    L5.32:
    A polynomial \(a(x) \in F[x]\) of degree at most \(d\) is uniquely
    determined by any \(d + 1\) values of \(a(x)\).

    \subsection{Finite fields}

    L5.33:
    Congruence modulo \(m(x)\) is an equivalence relation on \(F[x]\),
    and each equivalence class has a unique representation of degree
    less than \(\deg(m(x))\).

    D5.34:
    Let \(m(x)\) be a polynomial of degree \(d\) over \(F\). Then
    \(F[x]_{m(x)} \defeq \{a(x) \in F[x] \mid \deg(a(x)) < d\}\).

    L5.34:
    Let \(F\) be a finite field with \(q\) elements and let \(m(x)\) be a
    polynomial of degree \(d\) over \(F\). Then \(|F[x]_{m(x)}| = q^d\).

    L5.35:
    \(F[x]_{m(x)}\) is a ring with respect to addition and multiplication
    modulo \(m(x)\).

    L5.36:
    The congruence equation \(a(x) b(x) \equiv_{m(x)} 1\)
    has a solution \(b(x) \in F[x]_{m(x)}\) iff \(\gcd(a(x), m(x)) = 1\).
    The solution is unique. In other words,
    \(F[x]^*_{m(x)} = \{a(x) \in F[x]_{m(x)} \mid \gcd(a(x), b(x)) = 1\}\).

    T5.37:
    The ring \(F[x]_{m(x)}\) is a field iff \(m(x)\) is irreducible.

    \subsection{Application: Error-Correcting Codes}

    D5.35: An (\(n\), \(k\))-\textit{encoding function} \(E\) for some
    alphabet \(\mathcal A\) is an injective function that maps a list
    \(a_0, \ldots, a_{k-1} \in \mathcal A^k\) of \(k\) symbols
    to a list of \(c_0, \ldots, c_{n-1} \in \mathcal A^n\) of \(n > k\)
    (encoded) symbols in \(\mathcal A\) called \textit{codeword}.

    D5.36:
    An (\(n\), \(k\))-\textit{error-correcting code} over the alphabet
    \(\mathcal A\) with \(|\mathcal A| = q\) is a subset of \(\mathcal A^n\)
    of cardinality \(q^k\).

    D5.37:
    The \textit{Hamming distance} between two strings of equal
    length over a finite alphabet \(\mathcal A\) is the number of positions
    at which the two strings differ.

    D5.38:
    The \textit{minimum distance} of an error-correcting code \(\mathcal C\),
    denoted \(d_{\min(\mathcal C)}\), is the minimum of the Hamming distance
    between any two codewords.

    D5.39:
    A \textit{decoding function} \(D\) for an (\(n\), \(k\))-encoding function
    is a function \(D \colon \mathcal A^n \to \mathcal A^k\).

    D5.40:
    A decoding function \(D\) is \(t\)-\textit{error correcting} for the
    encoding function \(E\) if for any \((a_0, \ldots, a_{k-1})\) we have
    \(D((r_0, \ldots, r_{n-1})) = (a_0, \ldots, a_{k-1})\)
    for any \((r_0, \ldots, r_{n-1})\) with Hamming distance at most \(t\)
    from \(E((a_0, \ldots, a_{k-1}))\). A code \(\mathcal C\) is
    \(t\)-\textit{error correcting} if there exist \(E\) and \(D\) with
    \(\mathcal C = \Img(E)\) where \(D\) is \(t\)-error correcting.

    T5.41:
    A code \(\mathcal C\) with minimum distance \(d\) is \(t\)-error-correcting
    iff \(d \ge 2t + 1\).

    T5.42:
    Let \(\mathcal A = \GF(q)\) and let \(\alpha_0, \ldots, \alpha_{n-1}\) be
    arbitrary distinct elements of \(GF(q)\). Consider the encoding function
    \(E((a_0, \ldots, a_{k-1})) = (a(\alpha_0), \ldots, a(\alpha_{n-1}))\),
    where \(a(x)\) is the polynomial \(a(x) = a_{k-1}x^{k-1} + \cdots + a_1 x + a_0\).
    This code has a minimum distance of \(n - k + 1\).

    \section{Logic}

    \subsection{Proof systems}

    A \textit{proof system} is a quadruple \(\Pi = (\mathcal S, \mathcal P,
    \tau, \phi)\), where
    \begin{enumerate}
        \item \(\mathcal S, \mathcal P \subseteq \Sigma^*\),
        \item \(\tau \colon \mathcal S \to \{0, 1\}\),
        \item \(\tau \colon \mathcal S \times \mathcal P \to \{0, 1\}\),
    \end{enumerate}

    W.l.o.g. we consider \(\mathcal P = \mathcal S = \{0, 1\}^*\).

    D6.2:
    \(\Pi\) is \textit{sound} iff for all \(s \in \mathcal S\) for which there
    exists \(p \in \mathcal P\) with \(\phi(s, p) = 1\) we have \(\tau(s) = 1\).

    D6.3:
    \(\Pi\) is \textit{complete} iff for all \(s \in \mathcal S\) with
    \(\tau(s) = 1\) there exists \(p \in \mathcal P\) with
    \(\phi(s, p) = 1\).

    \subsection{Elementary general concepts in logic}

    D6.4:
    The \textit{syntax} of a logic defines an alphabet \(\Lambda\)
    and specifies which strings in \(\Lambda^*\) are formulas.

    D6.5:
    The \textit{semantics} of a logic defines a function \textit{free}
    which assigns to each formula \(F = (f_1, \ldots, f_k) \in \Lambda^*\)
    a subset \textit{free}\((F) \subseteq [k]\) of the indices. If
    \(i \in\) \textit{free}\((F)\), then the symbol \(f_i\) is said to occur
    \textit{free} in \(F\).

    D6.6:
    An \textit{interpretation} consists of a set \(\mathcal Z \subseteq \Lambda\),
    a domain for each symbol in \(\mathcal Z\), and a function that assigns
    that assigns to each symbol in \(\mathcal Z\) a value in its associated domain.

    D6.7:
    An interpretation is \textit{suitable} for a formula \(F\) if it assigns
    a value to all symbols \(\beta \in \Lambda\) occurring free in \(F\).

    D6.8:
    The \textit{semantics} of a logic also defines a function \(\sigma\)
    assigning to each formula \(F\) and each interpretation \(\mathcal A\)
    suitable for \(F\), a truth value \(\sigma(F, \mathcal A)\) in
    \(\{0, 1\}\). One often writes \(\mathcal A(F)\) instead of
    \(\sigma(F, \mathcal A)\) and calls \(\mathcal A(F)\) the
    \textit{truth value of} \(F\) \textit{under the interpretation}
    \(\mathcal A\).

    D6.9:
    A (suitable) interpretation \(\mathcal A\) for which the formula \(F\)
    is true is called a \textit{model} for \(F\) and one writes
    \(\mathcal A \models F\).
    % + trivial

    D6.10:
    \(F\) is called \textit{satisfiable} iff there exists a model for \(F\),
    and \textit{unsatisfiable} (denoted \(\bot\)) otherwise.

    D6.11:
    \(F\) is called a \textit{tautology} (denoted \(\top\)) iff it is true
    for every suitable interpretation.

    D6.12:
    A formula \(G\) is a \textit{logical consequence} of a formula \(F\),
    denoted \(F \models G\), if every interpretation suitable for both
    \(F\) and \(G\), which is a model for \(F\) is also a model for \(G\).

    D6.13 (Equivalence):
    \(F \equiv G \defiff F \models G\) and \(G \models F\).

    D6.14:
    If \(F\) is a tautology we write \(\models F\).

    D5.15:
    If \(F, G\) are formulas then also \(\lnot F\), \(F(F \land G)\),
    and \((F \lor G)\) are formulas.

    D6.16:

    \begin{align*}
        \mathcal A((F \land G)) = 1 &\iff \mathcal A(F) = 1 \text{ and } \mathcal A(G) = 1 \\
        \mathcal A((F \lor G)) = 1 &\iff \mathcal A(F) = 1 \text{ or } \mathcal A(G) = 1 \\
        \mathcal A(\lnot F) = 1 &\iff \mathcal A(F) = 0
    \end{align*}

    L6.1:
    For any formulas \(F\), \(G\) and \(H\) we have
    \begin{center}
        \begin{tabular}{|l|l|}
            \hline
            name & law \\
            \hline
            \textit{idempot.} & \(F \stackrel{\lor}{\land} F \equiv F\) \\
            \textit{commut.} & \(F \stackrel{\lor}{\land} G = G \stackrel{\lor}{\land} F\) \\
            \textit{assoc.} & \(F \stackrel{\lor}{\land} (G \stackrel{\lor}{\land} H) =
                (F \stackrel{\lor}{\land} G) \stackrel{\lor}{\land} H\) \\
            \textit{absorp.} & \(F \stackrel{\lor}{\land} (F \stackrel{\land}{\lor} G) = F\) \\
            \textit{distrib.} & \(F \stackrel{\lor}{\land} (G \stackrel{\land}{\lor} H) =
            (F \stackrel{\lor}{\land} G) \stackrel{\land}{\lor} (F \stackrel{\lor}{\land} H)\) \\
            \textit{double neg.} & \(\lnot \lnot F \equiv F\) \\
            \textit{deMorgan} & \(\lnot (F \stackrel{\lor}{\land} G) \equiv \lnot F \stackrel{\land}{\lor} \lnot G\) \\
            \textit{taut.} & \(F \lor \top \equiv \top\) and \(F \land \top \equiv F\) \\
            \textit{unsat.} & \(F \lor \bot \equiv F\) and \(F \land \bot \equiv \bot\) \\
            \textit{} &  \(F \lor \lnot F \equiv \top\) and \(F \land \lnot F \equiv \bot\) \\
            \hline
        \end{tabular}
    \end{center}

    L6.2:
    \(F\) is a tautology iff \(\lnot F\) is unsatisfiable.

    L6.3:
    The following statements are equivalent:
    \begin{enumerate}
        \item \(\{F_1, \ldots, F_k\} \models G\),
        \item \((F_1 \land \ldots \land F_k) \to G\) is a tautology,
        \item \(\{F_1, \ldots, F_k, \lnot G\}\) is unsatisfiable.
    \end{enumerate}

    \subsection{Logical calculi}

    D.17:
    A \textit{derivation rule} \(R\) is a relation from the power set of the set
    of formulas to the set of formulas and the symbol \(\vdash_R\) is the
    relation symbol.

    D6.18:
    The \textit{application of a derivation rule} \(R\) to a set \(M\) of
    formulas means
    \begin{enumerate}
        \item Select \(N \subseteq M\) such that \(N \vdash_R G\).
        \item Replace \(M\) with \(M \cup \{G\}\).
    \end{enumerate}

    D6.19:
    A (logical) \textit{calculus} \(K\) is a finite set of derivation rules
    \(K = \{R_1, \ldots, R_m\}\).

    D6.20:
    A \textit{derivation} of a formula \(G\) from a set \(M\) of
    formulas in a calculus \(K\) is a finite sequence of applications of rules
    in \(K\), leading to \(G\).
    We write \(M \vdash_K G\).

    D6.21:
    \(R\) is \textit{correct} iff
    \(M \vdash_R F \implies M \models F\).

    D6.22:
    \(K\) is \textit{sound} iff \(M \vdash_K F \implies M \models F\). \\
    \(K\) is \textit{complete} iff \(M \models F \implies M \vdash_K F\).

    \subsection{Propositional logic}

    D6.23 (Syntax):
    An \textit{atomic formula} is a symbol of the form \(A_i\).
    A \textit{formula} is defined as follows:
    \begin{enumerate}
        \item An atomic formula is a formula.
        \item See [D6.15].
    \end{enumerate}

    D6.24 (Semantics):
    Each atomic formula is assigned a truth value.
    Then see [D6.16].

    D6.25:
    A \textit{literal} is an atomic formula or the negation of an atomic formula.

    D6.26:
    A formula \(F\) is in \textit{conjunctive normal form} (CNF)
    iff it is of the form
    \(F = (A \lor \cdots \lor B) \land \cdots \land (Y \lor \cdots \lor Z)\).

    D6.27:
    A formula \(F\) is in \textit{disjunctive normal form} (DNF)
    iff it is of the form
    \(F = (A \land \cdots \land B) \lor \cdots \lor (Y \land \cdots \land Z)\).

    T6.4:
    Every formula is equivalent to a formula in CNF and a formula in DNF.

    D6.28:
    A \textit{clause} is a set of literals.

    D6.29:
    The set of clauses associated to a formula in CNF (see [D6.26]) is
    \(\mathcal K(F) \defeq \{\{A, \ldots, B\}, \ldots, \{Y, \ldots, Z\}\}\).

    D6.30:
    The clause \(K\) is a \textit{resolvent} of clause \(K_1\) and \(K_2\)
    if there is a literal \(L\) such that \(L \in K_1\), \(\lnot L \in K_2\),
    and \(K = (K_1 \setminus \{L\}) \cup (K_2 \setminus \{\lnot L\})\).

    D (Resolution calculus):
    Let \(K_1, K_2, K\) as in [D6.30].
    Then the \textit{resolution rule} is \(\{K_1, K_2\} \vdash_{\res} K\)
    and the \textit{reolution calculus} is \(\RES = \{ \res \}\).

    L6.5:
    The resolution calculus is sound.

    T6.6:
    A set \(M\) of formulas is unsat. iff \(\mathcal K(M) \vdash_{\Res} \varnothing\).

    \subsection{Predicate logic}

    D6.31 (syntax of predicate logic):
    \begin{enumerate}
        \item \textit{variable}: \(x_i\)
        \item \textit{function}: \(f_i^{(k)}\)
        \item \textit{predicate}: \(P_i^{(k)}\)
        \item \textit{term}: Variables are terms and if \(t_1, \ldots, t_k\) are terms, then
            \(f_i^{(k)}(t_1, \ldots, t_k)\) is a term.
        \item \textit{formula}: \begin{itemize}
            \item If \(t_1, \ldots, t_k\) are terms, then \(P_i^{(k)}(t_1, \ldots, t_k)\)
                is an \textit{atomic} formula.
            \item If \(F\) and \(G\) are formulas, then \(\lnot F\),
                \((F \land G)\), \((F \lor G)\) are formulas.
            \item If \(F\) is a formula then \(\forall x_i F\) and
                \(\exists x_i F\) are formulas.
        \end{itemize}
    \end{enumerate}
    

    D6.32:
    Every occurence of a variable in a formula is either \textit{bound} or
    \textit{free}. Iff a variable \(x\) occurs in a (sub-)formula of the form
    \(\forall x G\) or \(\exists x G\) then it is bound. A formula is
    \textit{closed} if it contains no free variables.

    D6.33:
    For a formula \(F\), a variable \(x\) and term \(t\), \(F[x \slash t]\)
    denotes the formula obtained from \(F\) by substituting every free
    occurence of \(x\) by \(t\).

    D6.34:
    An \textit{interpretation} is a tuple \(\mathcal A = (U, \phi, \psi, \xi)\)
    where
    \begin{itemize}
        \item \(U\) is a non-empty \textit{universe},
        \item \(\phi\) is a function assigning to each function symbol
            a function, \(\phi(f^{(k)}) : U^k \to U\),
        \item \(\psi\) is a function assigning to each predicate symbol
            a function, \(\phi(P^{(k)}) : U^k \to \{0, 1\}\),
        \item \(\xi\) is a function assigning to each variable symbol
            a value, \(\phi(x) \in U\).
    \end{itemize}

    D6.35:
    An interpretation \(\mathcal A\) is \textit{suitable} for a formula \(F\)
    iff it defines all function symbols, predicate symbols and freely occurring
    variables of \(F\).

    D6.36 (semantics):
    For an interpretation \(\mathcal A = (U, \phi, \psi, \xi)\),
    we define the value (in \(U\)) of terms and the truth value of formulas as follows:

    1. The value \(\mathcal A(t)\) of a term \(t\) is defined recursively:
            \begin{itemize}
                \item If \(t = x_i\), then \(\mathcal A(t) = \xi(x_i)\).
                \item If \(t = f(t_1, \ldots, t_k)\), then
                    \(\mathcal A(t) = \phi(f)(\mathcal A(t_1), \ldots, \mathcal A(t_k))\).
            \end{itemize}
    2. The truth value of a formula \(F\) is defined recursively:
            \begin{itemize}
                \item See [D6.16].
                \item If \(F = P(t_1, \ldots, t_k)\), then
                    \(\mathcal A(F) = \psi(P)(\mathcal A(t_1), \ldots, \mathcal A(t_k))\).
                \item \(\mathcal A(\forall x G) = \begin{cases}
                    1 & \mathcal A_{[x \to u]}(G) = 1 \text{ for all } u \in U \\
                    0 & \text{otherwise .} 
                \end{cases}\)
                \item \(\mathcal A(\exists x G) = \begin{cases}
                    1 & \mathcal A_{[x \to u]}(G) = 1 \text{ for some } u \in U \\
                    0 & \text{otherwise .} 
                \end{cases}\)
            \end{itemize}

    L6.7:
    For any formulas \(F\), \(G\), \(H\), where \(x\) does not occur free in \(H\),
    we have
    \begin{enumerate}
        \item \(\lnot (\forall x F) \equiv \exists x \lnot F\);
        \item \(\lnot (\exists x F) \equiv \forall x \lnot F\);
        \item \((\forall x F) \land (\forall x G) \equiv \forall x (F \land G)\);
        \item \((\exists x F) \lor (\exists x G) \equiv \exists x (F \lor G)\);
        \item \(\forall x \forall y F \equiv \forall y \forall x F\);
        \item \(\exists x \exists y F \equiv \exists y \exists x F\);
        \item \((\forall x F) \land H \equiv \forall x (F \land H)\);
        \item \((\forall x F) \lor H \equiv \forall x (F \lor H)\);
        \item \((\exists x F) \land H \equiv \exists x (F \land H)\);
        \item \((\exists x F) \lor H \equiv \exists x (F \lor H)\).
    \end{enumerate}

    L6.8:
    If one replaces a sub-formula \(G\) of a formula \(F\) by an equivalent
    (to \(G\)) formula \(H\), then the resulting formula is equivalent to \(F\).

    L6.9:
    For a formula \(G\) in which \(y\) does not occur, we have:
    \begin{enumerate}
        \item \(\forall x G \equiv \forall y G[x \slash y]\),
        \item \(\exists x G \equiv \exists y G[x \slash y]\).
    \end{enumerate}

    D6.37:
    A formula in which no variable occurs both as a bound and as a free
    variable and in which all variables appearing after the quantifies are
    distinct is said to be in \textit{rectified} form.

    D6.38:
    A formula of the form \(Q_1 x_1 \cdots Q_n x_n G\), where \(Q_i\)
    are arbitrary quantifiers and \(G\) is a formula free of quantifiers
    is said to be in \textit{prenex form}.

    T6.10:
    For every formula there is an equivalent formula in prenex form.

    L6.11:
    For any formula \(F\) and any term \(t\) we have
    \(\forall x F \models F[x \slash t]\).

    T6.12:
    \(\lnot \exists x \forall y (P(y, x) \lrarr \lnot P(y, y))\).

    % C5.13:
    % \(\{S \mid S \not\in S\}\) is not a set.

    % C5.15:
    % There are uncomputable functions \(\mathbb N \to \{0, 1\}\).

    % C5.16:
    % The function \(\mathbb N \to \{0, 1\}\), assigning to each
    % \(y \in \mathbb N\) the complement of what program \(y\) outputs
    % on input \(y\), is uncomputable.

    \section{Lookup}

    \subsection{Factorizations}

    \(2023 = 7 \cdot 17^2 \quad 2024 = 2^3 \cdot 11 \cdot 23 \quad 2025 = 3^4 \cdot 5^2\)

    \subsection{Small primes}

    2, 3, 5, 7, 11, 13, 17, 19, 23, 29, 31, 37, 41, 43, 47, 53, 59, 61, 67, 71, 73, 79, 83, 89, 97, 101, 103, 107, 109, 113, 127, 131, 137, 139, 149, 151, 157, 163, 167, 173, 179, 181, 191, 193, 197, 199, 211, 223, 227, 229, 233, 239, 241, 251, 257, 263, 269, 271, 277, 281, 283, 293, 307, 311, 313, 317, 331, 337, 347, 349, 353, 359, 367, 373, 379, 383, 389, 397, 401, 409, 419, 421, 431, 433, 439, 443, 449, 457, 461, 463, 467, 479, 487, 491, 499, 503, 509, 521, 523, 541, 547, 557, 563, 569, 571, 577, 587, 593, 599, 601, 607, 613, 617, 619, 631, 641, 643, 647, 653, 659, 661, 673, 677, 683, 691, 701, 709, 719, 727, 733, 739, 743, 751, 757, 761, 769, 773, 787, 797, 809, 811, 821, 823, 827, 829, 839, 853, 857, 859, 863, 877, 881, 883, 887, 907, 911, 919, 929, 937, 941, 947, 953, 967, 971, 977, 983, 991, 997, 1009, 1013, 1019, 1021, 1031, 1033, 1039, 1049, 1051, 1061, 1063, 1069, 1087, 1091, 1093, 1097, 1103, 1109, 1117, 1123, 1129, 1151, 1153, 1163, 1171, 1181, 1187, 1193, 1201, 1213, 1217, 1223, 1229, 1231, 1237, 1249, 1259, 1277, 1279, 1283, 1289, 1291, 1297, 1301, 1303, 1307, 1319, 1321    

    \subsection{Small groups}

    \begin{center}
        \begin{tabular}{|l|l|l|}
            \hline
            \(|G|\) & abelian & non-abelian \\
            \hline
            1 & \(Z_1\) & \\
            \hline
            2 & \(Z_2\) & \\
            \hline
            3 & \(Z_3\) & \\
            \hline
            4 & \(Z_4, Z_2^2\) & \\
            \hline
            5 & \(Z_5\) & \\
            \hline
            6 & \(Z_6\) & \(D_6\) \\
            \hline
            7 & \(Z_7\) & \\
            \hline
            8 & \(Z_8, Z_4 \times Z_2, Z_2^3\) & \(D_8, Q_8\) \\
            \hline
            9 & \(Z_9, Z_3^2\) & \\
            \hline
            10 & \(Z_{10}\) & \(D_{10}\) \\
            \hline
        \end{tabular}
    \end{center}
    
    \subsection{Euler function}

    \begin{center}
        \begin{tabular}{|l|l|l|l|l|l|l|l|l|l|}
            \hline
            \(x\) & \(\varphi\) & \(x\) & \(\varphi\) & \(x\) & \(\varphi\) & \(x\) & \(\varphi\) \\
            \hline
            1 & 1   & 21 & 12 & 41 & 40 & 61 & 60 \\
            \hline
            2 & 1   & 22 & 10 & 42 & 12 & 62 & 30 \\
            \hline
            3 & 2   & 23 & 22 & 43 & 42 & 63 & 36 \\
            \hline
            4 & 2   & 24 & 8  & 44 & 20 & 64 & 32 \\
            \hline
            5 & 4   & 25 & 20 & 45 & 24 & 65 & 48 \\
            \hline
            6 & 2   & 26 & 12 & 46 & 22 & 66 & 20 \\
            \hline
            7 & 6   & 27 & 18 & 47 & 46 & 67 & 66 \\
            \hline
            8 & 4   & 28 & 12 & 48 & 16 & 68 & 32 \\
            \hline
            9 & 6   & 29 & 28 & 49 & 42 & 69 & 44 \\
            \hline
            10 & 4  & 30 & 8  & 50 & 20 & 70 & 24 \\
            \hline
            11 & 10 & 31 & 30 & 51 & 32 & 71 & 70 \\
            \hline
            12 & 4  & 32 & 16 & 52 & 24 & 72 & 24 \\
            \hline
            13 & 12 & 33 & 20 & 53 & 52 & 73 & 72 \\
            \hline
            14 & 6  & 34 & 16 & 54 & 18 & 74 & 36 \\
            \hline
            15 & 8  & 35 & 24 & 55 & 40 & 75 & 40 \\
            \hline
            16 & 8  & 36 & 12 & 56 & 24 & 76 & 36 \\
            \hline
            17 & 16 & 37 & 36 & 57 & 36 & 77 & 60 \\
            \hline
            18 & 6  & 38 & 18 & 58 & 28 & 78 & 24 \\
            \hline
            19 & 18 & 39 & 24 & 59 & 58 & 79 & 78 \\
            \hline
            20 & 8  & 40 & 16 & 60 & 16 & 80 & 32 \\
            \hline
        \end{tabular}
    \end{center}

    \newpage

    \section{Appendix}

    \subsection{Notation}

    R:
    The word \textit{iff} stands for "if and only if".

    D:
    Let \(n \in \mathbb Z^+\). Then \([n] := \{1, \ldots, n\}\).

    \subsection{Symbols}

    \begin{center}
        \begin{tabular}{|l|l|}
            \hline
            A & Algorithm \\
            \hline
            C & Corollary \\
            \hline
            D & Definition \\
            \hline
            E & Example \\
            \hline
            F & Fact \\
            \hline
            L & Lemma \\
            \hline
            O & Observation \\
            \hline
            P & Proposition \\
            \hline
            R & Remark \\
            \hline
            T & Theorem \\
            \hline
        \end{tabular}
    \end{center}

    % further layouts

    % two-column layout within the section
    % \begin{minipage}{0.49\linewidth}
    %     Left
    % \end{minipage}
    % \hfill
    % \begin{minipage}{0.49\linewidth}
    %    Right
    % \end{minipage}

    % \section{Boxed Equation}
    % % Boxed equation with a simple math expression
    % \mathbox{
    %     $E = mc^2$
    % }

    % \section{Input}
    % Including an external file
    % \input{sections/00_Example}

    % \section{Image}
    % Including an image with full width
    % \includegraphics[width=\linewidth]{img/example.png} 
}
