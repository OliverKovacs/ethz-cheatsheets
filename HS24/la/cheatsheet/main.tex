\documentclass[a4paper, landscape]{article}
\usepackage[top=1.7cm, bottom=0.1cm, left=0.5cm, right=0.5cm, headheight=1cm, footskip=0.3cm]{geometry}
\usepackage{fancyhdr} % Customizable headers and footers
\usepackage{multicol} % Allows for multiple columns
\usepackage[utf8]{inputenc} % Ensures UTF-8 encoding
\usepackage[ngerman]{babel} % Language support for German
\usepackage{amsmath, amsfonts} % Math packages for symbols and fonts
\usepackage{lmodern} % Enhances font quality
\usepackage{graphicx} % For including images
\usepackage[normalem]{ulem} % Provides underlining capabilities
\usepackage[dvipsnames, table]{xcolor} % Adds color options, including for tables
\usepackage{enumitem} % Customizable lists
\usepackage{mathabx} % Additional math symbols
\usepackage{colortbl} % Color options for tables
\usepackage{mathtools} % Additional tools for mathematical typesetting
\usepackage{wallpaper} % Allows adding wallpaper backgrounds
\usepackage{changepage} % Enables adjustments to margins
\usepackage{tikz} % Graphics for drawing and illustrations
\usepackage{tabularx} % Advanced table formatting
\usepackage[skins]{tcolorbox} % Colored boxes with advanced options
\usepackage{lipsum} % For generating dummy text
\usepackage{bbm} % Font package, especially for blackboard bold characters
\usepackage{multirow} % Combines rows in tables
\usepackage{letltxmacro} % Advanced command management
\usepackage{float} % Control float placement
\usepackage{amssymb} % Additional math symbols
\usepackage{algorithm} % Algorithm environment
\usepackage[noend]{algpseudocode} % Pseudocode environment
\usepackage{leftidx} % Left sub/superscripts
\usepackage{empheq} % Emphasized equations
\usepackage{textcomp} % Additional text symbols
\usepackage{calc} % Arithmetic operations within LaTeX lengths
\usepackage[makeroom]{cancel} % Strike through terms in equations
\usepackage{sidecap} % Places captions to the side of figures
\usepackage{physics} % Additional physics-related commands (e.g., Dirac notation)
\usepackage{makecell} % More advanced cell formatting in tables

\usepackage{mathrsfs}

% Custom header format using fancyhdr package
\pagestyle{fancy}
\fancyhf{} % Clears default header/footer
\lhead{\textbf{\subject} - \semester} % Left header
\rhead{\author} % Right header
\fancyfoot[C]{Page - \thepage} % Center the page number at the bottom

% Custom bullet symbol for itemized lists
\renewcommand\textbullet{\ensuremath{\bullet}}

% Command to create circled numbers for lists
\newcommand*\circled[1]{\tikz[baseline=(char.base)]{
        \node[shape=circle,draw,inner sep=1.2pt] (char) {#1};}}

% Increase row height in tables
\renewcommand{\arraystretch}{1.5}

\newtcbox{\mathbox}[1][]{
    nobeforeafter, 
    colframe=white!30!black, % Frame color
    colback=white!20, % Background color
    boxrule=0.5pt, % Border thickness
    arc=0mm, % Rounded corners
    before skip = 1mm,
    right = 1mm,
    left=1mm,
    top=0.5mm,
    bottom=0.5mm,
    center, % Center the content
    #1
}

% Fill the remaining space with blank space (in multicols environment)
\newcommand{\fillblank}{
    \vfill\null\columnbreak
}

% Set the default font family to sans-serif
\renewcommand{\familydefault}{\sfdefault}

% Custom command for creating a cheatsheet environment
\newcommand{\cheatsheet}[1]{
    \let\underbrace\LaTeXunderbrace % Fixes issue with underbrace in custom environments
    \let\overbrace\LaTeXoverbrace % Fixes issue with overbrace in custom environments
    \begin{document}
    \setlength{\columnseprule}{0.4pt} % Sets the rule width between columns
    \footnotesize
    \begin{multicols*}{\cols} % Starts a multicolumn environment
        \setlength{\abovedisplayskip}{\abovedisplayskip - 4mm} % Reduces space above displayed equations
        \setlength{\belowdisplayskip}{\belowdisplayskip - 3mm} % Reduces space below displayed equations
    #1
    \end{multicols*}
    \end{document}
}

% Adjust the space between paragraphs
\setlength{\parskip}{0.1cm}

% Custom section formatting
\renewcommand{\section}[1]{
    \vskip 1pt
    \par\noindent\textbf{\normalsize#1} % Section title in bold, normal size
    \vskip -8pt
    \par\noindent\smash{\rule[-1.5pt]{2cm}{2pt}}\rule{\linewidth  - 2cm}{0.5pt} % Custom underline
    \par\vspace{-0.5mm}
}

% Custom subsection formatting
\renewcommand{\subsection}[2][]{
    \vskip 1pt
    \par\noindent\textbf{\small#2} \textbf{\small#1} % Subsection title in bold, small size
    \vskip -8pt
    \par\noindent\smash{\rule[-1.5pt]{1cm}{2pt}}\rule{\linewidth - 1cm}{0.5pt} % Custom underline
    \par\vspace{-1mm}
}

% Custom subsubsection formatting
\renewcommand{\subsubsection}[1]{
    \vskip 1pt
    \par\noindent\textbf{\footnotesize#1} % Subsubsection title in bold, footnote size
    \vskip -8pt
    \rule{\linewidth}{0.5pt} % Horizontal rule across the width
    \par\vspace{-1mm}
}

% No paragraph indentation
\setlength{\parindent}{0pt}
% Column settings for multicol environment
\setlength\columnsep{2mm}
\setlength{\columnseprule}{0pt}
\setlength{\fboxrule}{0.5pt}
\setlength{\headsep}{0.5em}
\setlength{\footskip}{12.0pt}



\newcommand{\sgn}{\operatorname{sgn}}
\newcommand{\proj}{\operatorname{proj}}
\newcommand{\Span}{\operatorname{Span}}
\newcommand{\Sol}{\operatorname{Sol}}
\newcommand{\Ker}{\operatorname{Ker}}
\newcommand{\Im}{\operatorname{Im}}
\newcommand{\REF}{\operatorname{REF}}
\newcommand{\RREF}{\operatorname{RREF}}


% global setttings
\def\subject{Lineare Algebra}
\def\semester{ETH Zürich (401-0131-00L, HS24)}
\def\author{Olivér Kovács}
\def\cols{4}

\cheatsheet{
    \section{Vectors}

    \subsection{Linear combinations}

    % D1.1:
    % Vector addition.

    % D1.3:
    % Scalar multiplication.

    % D1.4:
    % Linear combinations.

    D1.7:
    A linear combination \(\lambda_1 v_1 + \cdots + \lambda_n v_n\)
    is called
    \begin{enumerate}
        \item \textit{affine} if \(\lambda_1 + \cdots + \lambda_n = 1\),
        \item \textit{conic} if \(\lambda_1, \ldots, \lambda_n \ge 0\),
        \item \textit{convex} if it is both affine and conic.
    \end{enumerate}

    \subsection{Scalar products, lengths, angles}

    % D1.9:
    % Scalar product.

    O1.10:
    Let \(u, v, w \in \mathbb R^m, \lambda \in \mathbb R\).
    Then
    \begin{enumerate}
        \item \(v \cdot w = w \cdot v\);
        \item \((\lambda v) \cdot w = \lambda(v \cdot w) = v \cdot (\lambda w)\);
        \item \(u \cdot (v + w) = u \cdot v + u \cdot w\) and 
            \((u + v) \cdot w = u \cdot v + v \cdot w\);
        \item \(v \cdot v \ge 0\), equality iff \(v = \mathbf 0\).
    \end{enumerate}

    D1.11 (Euclidean norm):
    Let \(v \in \mathbb R^m\).
    Then \(\|v\| := \sqrt{v \cdot v}\).

    L1.12 (Cauchy-Schwarz inequality):
    Let \(v, w \in \mathbb R^m\).
    Then \(|v \cdot w| \le \|v\|\|w\|\).
    Equivalently \((uw)^2 \le v^2 w^2\).
    Equality iff \(v = \lambda w\).

    D1.14 (Angle):
    Let \(v, w \in \mathbb R^m\) nonzero. Then
    \(\cos(\alpha) = \frac{v \cdot w}{\|v\| \|w\|}\).

    D1.15:
    Two vectors \(v, w \in \mathbb R^m\) are called perpendicular/orthogonal
    iff \(v \cdot w = 0\).

    L1.16:
    Let \(v, w \in \mathbb R^m\).
    Then \(\|v + w\| \le \|v\| + \|w\|\).

    \subsection{Linear independence}

    % D1.18:
    % Linear independence.

    L1.19:
    Let \(v_1, \ldots, v_n \in \mathbb R^n\).
    The following statements are equivalent:
    \begin{enumerate}
        \item At least one of the vectors is a linear combination of the other ones.
        \item \(\mathbf 0\) is nontrivial linear combination of the vectors.
        \item At least one of the vectors is a linear combination of the prevous ones.
    \end{enumerate}

    % C1.20:
    % L1.19 inverted.

    L1.21:
    Let \(v_1, \ldots, v_n \in \mathbb R^m\) be linearly independent and
    \(\sum_{j=1}^n \lambda_j v_j = \sum_{j=1}^n \mu_j v_j\).
    Then \(\lambda_j = \mu_j\) for all \(j \in [n]\).

    D1.22 (Span):
    Let \(v_1, \ldots, v_n \in \mathbb R^m\). Then
    \(\Span(v_1, \ldots, v_n) := \{\sum_{j=1}^n \lambda_j v_j : \lambda_j \in \mathbb R\}\).

    L1.23:
    Let \(v_1, \ldots, v_n \in \mathbb R^m\) and \(v \in \mathbb R^m\) a
    linear combination of \(v_1, \ldots v_n\). Then
    \(\Span(v_1, \ldots, v_n) = \Span(v_1, \ldots, v_n, v)\).

    \section{Matrices}

    \subsection{Linear combinations}

    % D2.1:
    % Matrix.

    % D2.2:
    % Matrix operations.

    D2.3:
    Let \(A = [a_{ij}]^{mm}_{i=1,j=1}\).
    If \(j < i, j = i, j > i\) then \(a_{ij}\) is \textit{below},
    \textit{on}, \textit{above} the diagonal.
    \begin{enumerate}
        \item If \(a_{ii} = 1\) and \(a_{ij} = 0\) for \(j \neq i\) then
            \(A = I\) is the \textit{identity} matrix.
        \item If \(a_{ij} = 0\) for \(j \neq i\) then
            \(A\) is \textit{diagonal}.
        \item If \(a_{ij} = 0\) for \(j < i\) then
            \(A\) is \textit{upper triangular}.
        \item If \(a_{ij} = 0\) for \(j > i\) then
            \(A\) is \textit{lower triangular}.
        \item If \(a_{ij} = a_{ji}\) for all \(i, j\) then
            \(A\) is \textit{symmetric}.
    \end{enumerate}

    % C2.6:
    % See [2.20].
    %Let \(I \in \mathbb R^{m \times m}, x \in \mathbb R^m\). Then \(Ix = x\).

    % D2.8:
    % See [L4.11].

    D2.9:
    Let \(A \in \mathbb R^{m \times n}\) with columns \(v_1, \ldots, v_n\).
    Column \(v_j\) is independent iff \(v_j\) is not a linear combination of
    \(v_1, \ldots, v_{j-1}\). \(\rank(A)\) is the number of independent columns.

    % L2.10:
    % [Trivial].

    D2.11:
    \(A^\top\) is the transpose of \(A\).

    O2.12:
    Let \(A \in \mathbb R^{m \times n}\). Then \((A^\top)^\top = A\).

    D2.13:
    Let \(A \in \mathbb R^{m \times n}\). Then \(R(A) := C(A^\top)\).

    % L2.14:
    % See [L2.21].

    \subsection{Matrix multiplication}

    D2.16:
    Let \(A \in \mathbb R^{a \times n}\) and \(B \in \mathbb R^{n \times b}\)
    with columns \(b_k\). Then \(AB \in \mathbb R^{a \times b}\) has columns
    \(Ab_k\).

    L2.19:
    Let \(A \in \mathbb R^{a \times n}, B \in \mathbb R^{n \times b}\).
    Then \((AB)^\top = B^\top A^\top\).

    C2.20:
    Let \(I \in \mathbb R^{m \times m}\).
    Then \(IA = A\) for \(A \in \mathbb R^{m \times n}\) and
    \(AI = A\) for \(A \in \mathbb R^{n \times m}\).

    L2.21:
    Let \(A \in \mathbb R^{m \times n}\).
    The following statements are equivalent:
    \begin{enumerate}
        \item \(\rank(A) = 1\).
        \item There are nonzero \(v \in \mathbb R^m, w \in \mathbb R^n\)
            such that \(A = vw^\top\).
    \end{enumerate}

    L2.22:
    Let \(A, B, C, D\) matrices sucht that sums and products are defined.
    Then
    \begin{enumerate}
        \item \(A(B + C) = AB + AC\) and \((B + C)D = BD + CD\)
        \item \((AB)C = A(BC)\).
    \end{enumerate}

    \subsubsection{CR decomposition}

    T2.23:
    Let \(A \in \mathbb R^{m \times n}, \rank(A) = r\).
    Let \(C \in \mathbb R^{m \times r}\) be the submatrix of \(A\)
    containing the independent columns. Then there exists a unique
    \(R \in \mathbb R^{r \times n}\) such that \(A = CR\).

    \subsection{Linear transformations}

    D2.25:
    Let \(A \in \mathbb R^{m \times n}\).
    \(T_A: \mathbb R^n \to \mathbb R^m\) is defined by
    \(T_A(x) = Ax\).

    O2.26:
    \(T_A\) is a linear transformation.

    D2.27 (Linear transformation):
    Let \(T: \mathbb R^n \to \mathbb R^m\).
    \(T\) is called a \textit{linear transformation} iff for all
    \(x, y \in \mathbb R^n, \lambda \in \mathbb R\)
    \begin{enumerate}
        \item \(T(x + y) = T(x) + T(y)\) and
        \item \(T(\lambda x) = \lambda T(x)\).
    \end{enumerate}

    % L2.28:
    % [Trivial]

    T2.29:
    Let \(T: \mathbb R^n \to \mathbb R^m\) be a linear transformation.
    There exists a unique \(A \in \mathbb R^{m \times n}\)
    such that \(T = T_A\).

    L2.30:
    Let \(T_A: \mathbb R^n \to \mathbb R^a, T_B: \mathbb R^b \to \mathbb R^n\)
    be linear transformations.
    Then \(T_A(T_B(x)) = T_{AB}(x)\).

    D2.31 (Kernel and image):
    Let \(T: \mathbb R^n \to \mathbb R^m\) be a linear transformation.
    Then
    \begin{enumerate}
        \item \(\Ker(T) := \{x \in \mathbb R^n : T(x) = \mathbf 0\} \subseteq \mathbb R^n\),
        \item \(\Img(T) := \{T(x) : x \in \mathbb R^n\} \subseteq \mathbb R^m\).
    \end{enumerate}

    O2.32:
    Let \(T: \mathbb R^n \to \mathbb R^m\) be a linear transformation and
    \(A \in \mathbb R^{m \times n}\) such that \(T = T_A\).
    Then \(\Img(T) = C(A)\).
    
    \section{Solving linear equations}

    \subsection{Systems of linear equations}

    % D3.1:
    % System of linear equations.

    O3.2:
    Let \(A \in \mathbb R^{m \times n}\). The columns of \(A\)
    are linearly independent iff \(Ax = \mathbf 0\) has a unique solution,
    \(x = \mathbf 0\).

    \subsection{Gauss elimination}

    A:
    Let \(A \in \mathbb R^{m \times m}\), \(b \in \mathbb R^m\). Then

    \begin{algorithmic}
        \Procedure{\textit{Gauss-elimination}}{$A, b$}
        \For{$j \gets 1, \ldots, m$}
            \If{$A_{j,j} = 0$}
                \State $k \gets j + 1$
                \While{$k \le m \land A_{k,j} = 0$}
                    $k \gets k + 1$
                \EndWhile
                \If{$k > m$}
                    \Return "gibs auf"
                \Else
                    \ exchange rows $j$ and $k$ in $A$, $b$
                \EndIf
            \EndIf

            \For{$i \gets j + 1, \ldots, m$}
                \State $c \gets A_{i,j} / A_{j,j}$
                \State subtract $c \ \cdot$ row $j$ from row $i$ in $A$, $b$
            \EndFor
        \EndFor
        \EndProcedure
    \end{algorithmic}

    L3.3:
    Let \(Ax = b\) be a system of \(m\) linear equations in \(n\) variables,
    \(M \in \mathbb R^{m \times m}\) be a row operation matrix.
    Let \(A' = MA, b' = Mb\).
    Then \(Ax = b\) and \(A'x = b'\) have the same solutions.

    % C3.4:
    % [Trivial]

    T3.5:
    The following statements are equivalent:
    \begin{enumerate}
        \item Gauss elimination succeeds.
        \item The columns of \(A\) are linearly independent.
    \end{enumerate}

    T3.6:
    Gauss elimination is in \(O(m^3)\).

    \subsection{Inverse matrices}

    D3.7:
    Let \(M \in \mathbb R^{m \times m}\). \(M\) is called \textit{invertible}
    iff there is an \(M^{-1} \in \mathbb R^{m \times m}\) such that
    \(M M^{-1} = M^{-1} M = I\).

    L3.8:
    The inverse of a matrix is unique.

    L3.9:
    Let \(A, B \in \mathbb R^{m \times m}\) be invertible.
    Then \((AB)^{-1} = B^{-1} A^{-1}\).

    L3.10:
    Let \(A \in \mathbb R^{m \times m}\) be invertible.
    Then \((A^\top)^{-1} = (A^{-1})^\top\).

    T3.11:
    Let \(A \in \mathbb R^{m \times m}\).
    The following statements are equivalent.
    \begin{enumerate}
        \item \(A\) is invertible.
        \item For every \(b \in \mathbb R^m\), \(Ax = b\) has a unique solution.
        \item The columns of \(A\) are linearly independent.
    \end{enumerate}

    \subsection{LU and LUP decomposition}

    T3.13:
    Let \(A \in \mathbb R^{n \times n}\) on which Gauss elimination
    succeeds without row exchanges, resulting in an upper triangular matrix
    \(U\). Let \(c_{ij}\) be the multiple of row \(j\) that we subtract from
    row \(i > j\) when we eliminate in column \(j\). Then \(A = LU\) where

    \[
        L = \begin{bmatrix}
            1       &        &           &   \\
            c_{2,1} &      1 &           &   \\
            \vdots  &        & \ddots    &   \\
            c_{m,1} & \cdots & c_{m,m-1} & 1 \\
        \end{bmatrix} .
    \]

    D3.14:
    A \textit{permutation} of \([m]\) is a bijective function
    \(\pi \colon [m] \to [m]\).
    
    D3.15:
    Let \(\pi \colon [m] \to [m]\) be a permutation.
    The \textit{permutation matrix} associated with \(\pi\) is
    \(P \in \mathbb Z^{m \times m}\) with \\
    \[
        p_{ij} =
        \begin{cases}
            1 & \text{if} j = \pi(i) \\
            0 & \text{otherwise}
        \end{cases} .
    \]

    L3.16:
    Let \(P\) be a permutation matrix. Then \(P^{-1} = P^\top\).

    L3.17:
    Let \(P, P' \in \mathbb Z^{n \times n}\) be permutation matrices with
    associated permutations \(\pi, \pi'\). Then \(PP'\) is a permutation matrix
    as well, associated with the permutation \(\pi' \circ \pi\).

    T3.18:
    Let \(A \in \mathbb R^{n \times n}\), \(m \ge 1\) have linearly
    independent columns. There exist \(P, L, U \in \mathbb R^{n \times n}\)
    such that \(PA = LU\) where \(P\) is a permutation matrix, \(L\) a lower triangular
    matrix with 1s on the diagonal and \(U\) an upper triangular matrix with nonzero
    diagonal entries.

    R (Solving \(Ax = b\) from \(PA = LU\)):
    Because \(P^{-1} = P^\top\) we have
    \(P^\top LUx = b\). Solve \(P^\top z = b\) for \(z\) be permutation.
    Solve \(Ly = z\) for \(y\) using forward substitution and \(Ux = y\) for \(x\)
    using backward substitution.

    \subsection{Gauss-Jordan elimination}

    D3.19 (REF, RREF):
    Let \(R \in \mathbb R^{m \times n}\).
    \(R\) is in row echelon form (REF) if the following holds.
    There exist \(r \le m\) column indices
    \(1 \le j_1 \le \cdots \le j_r \le n\) such that the following statements
    hold:
    \begin{enumerate}
        \item For \(i = 1, \ldots, r\) we have \(r_{i j_i} = 1\).
        \item For all \(i, j\) we have \(r_{ij} = 0\) whenever \(i > r\)
            or \(j < j_i\) or \(j = j_k\) for some \(k > i\).
    \end{enumerate}
    If \(r = m\), \(R\) is in reduced row echelon form (RREF).
    We use the notation \(\REF(j_1, \ldots, j_r)\) and
    \(\RREF(j_1, \ldots, j_r)\).

    O3.20:
    A matrix \(R\) in \(\REF(j_1, \ldots, j_r)\) has rank \(r\).

    A (Gauss-Jordan elimination):
    Like Gauss elimination, but:
    \begin{enumerate}
        \item Normalize pivot of each row to 1.
        \item Eliminate \textit{above} the pivot to get REF.
    \end{enumerate}

    T3.21 (Gauss-Jordan elimination):
    Let \(A \in \mathbb R^{m \times n}\). There exists an invertible matrix
    \(M \in \mathbb R^{m \times m}\) such that \(R_0 = MA\) is in REF.

    L3.22:
    Let \(A \in \mathbb R^{m \times n}\), \(M \in \mathbb R^{m \times m}\)
    invertible, and \(R_0 = MA\) in \(\REF(j_1, \ldots, j_r)\).
    Then \(A\) has independent columns \(j_1, \ldots, j_r\).

    T3.23:
    Let \(A \in \mathbb R^{m \times n}\), \(\rank(A) = r\) and
    \(b \in \mathbb R^m\).
    \begin{enumerate}
        \item Using Gauss-Jordan elimination, \(A\) can be transformed into
            \(R_0 = MA\) in REF as given by [T3.21] in time \(O(rmn + mn)\).
        \item By simultaneously transforming \(I \in \mathbb R^{m \times m}\)
            using the same row operations, \(M = MI\) can be computed in
            additional time \(O(rm^2 + m^2)\).
        \item Given \(M\), the system \(Ax = b\) can be solved in
            \(O(m^2)\).
    \end{enumerate}

    \subsubsection{Computing the CR decomposition}

    T3.24:
    Let \(A \in \mathbb R^{m \times n}\) and let \(A = CR\) as in [T2.23].
    Let \(R_0 = MA\) in \(\REF(j_1, \ldots, j_r)\) be the result of
    Gauss-Jordan elimination on \(A\) [T3.21].
    Then \(R\) results from \(R_0\) by removing the zero rows at the end
    (if there are any); in particular \(R\) is in \(\RREF(j_1, \ldots, j_r)\),
    and \(C\) is the submatrix of \(A\) with columns \(j_1, \ldots, j_r\).

    \section{The four fundamental subspaces}

    \subsection{Vector spaces}

    D4.1 (Vector space):
    A \textit{vector space} is a triple \((V, +, \cdot)\) where
    \(V\) is a set and
    \begin{enumerate}
        \item \(+: V \times V \to V\),
        \item \(\cdot: \mathbb R \times V \to V\),
    \end{enumerate}
    statisfying the following axioms for
    \(u, v, w \in V; \lambda, \mu \in \mathbb R\).
    \begin{enumerate}
        \item \(v + w = w + v\)
        \item \(u + (v + w) = (u + v) + w\)
        \item There is \(\mathbf 0 \in V\) such that \(v + \mathbf 0 = v\).
        \item There is \(\mathbf -v \in V\) such that \(v + (-v) = \mathbf 0\).
        \item \(1 \cdot v = v\)
        \item \((\lambda \mu) v = \lambda (\mu v)\)
        \item \(\lambda(v + w) = \lambda v + \lambda w\)
        \item \((\lambda + \mu)v = \lambda v + \mu v\)
    \end{enumerate}

    O4.2:
    \((\mathbb R^m, +, \cdot)\) is a vector space.

    D4.3:
    A \textit{polynomial} \(p\) is a sum of the form
    \(p = \sum_{i=0}^m p_i x^i\) for some \(m \in \mathbb N\).
    \(x\) is a \textit{variable} and \(p_0, \ldots, p_m \in \mathbb R\) are
    \textit{coefficients} of \(p\). The largest \(i\) such that
    \(p_i \neq 0\) is the \textit{degree} of \(p\).
    The zero polynomial \(\mathbf 0 = 0\) has degree -1.

    T4.4:
    Let \(\mathbb R[x]\) be the set of polynomials in \(x\).
    Given \(p = \sum_{i=0}^m p_i x^i\) and \(q = \sum_{i=0}^n q_i x^i\)
    and \(\lambda \in \mathbb R\).
    We define \(p + q = \sum_{i=0}^{\max(m, n)}(p_i + q_i)x^i\) and
    \(\lambda p = \sum_{i=0}^m(\lambda P_i) x^i\).
    Then \((\mathbb R[x], +, \cdot)\) is a vector space.

    T4.5:
    \((\mathbb R^{m \times n}, +, \cdot)\) is a vector space.

    F4.6:
    Each vector space contains exactly one zero vector.
    
    F4.7:
    Each \(v\) in a vector space has exactly one \(-v\).

    D4.8:
    Let \(V\) be a vector space. \(U \subseteq V\), \(U \neq \emptyset\)
    is called a subspace of \(V\) iff for all \(v, w \in \mathbb U\) and
    \(\lambda \in \mathbb R\).
    \begin{enumerate}
        \item \(v + w \in U\);
        \item \(\lambda v \in U\).
    \end{enumerate}

    L4.9:
    Let \(U \subseteq V\) be a subspace. Then \(\mathbf 0 \in V\).

    L4.11:
    Let \(A \in \mathbb R^{m \times n}\). Then
    \(C(A) = \{Ax \colon x \in \mathbb R^n\}\) is a subspace of \(\mathbb R^m\).

    L4.12:
    Let \(V\) be a vector space and \(U\) a subspace.
    Then \(U\) is a vector space.

    \subsection{Bases and dimension}

    % D4.13:
    % (Linear combination of a set of vectors).

    L4.14:
    Let \(V\) be a vector space \(G \subseteq V\).
    Every linear combination of \(G\) is in \(V\).

    D4.16 (Basis):
    Let \(V\) be a vector space.
    \(B \subseteq V\) is called a basis of \(V\) iff \(B\) is linearly independent
    and \(\Span(B) = V\).

    O4.18:
    Every set of \(m\) linearly independent vectors is a basis of \(\mathbb R^m\).

    L4.19 (Steinitz exchange lemma):
    Let \(V\) be a vector space, \(F \subseteq V\) finite and
    linearly independent and \(G \subseteq V\) finite with \(\Span(G) = V\).
    Then
    \begin{enumerate}
        \item \(|F| \le |G|\)
        \item There exists \(E \subseteq G\) of size \(|G| - |F|\) such that
            \(\Span(F \cup E) = V\).
    \end{enumerate}

    T4.20:
    Let \(V\) be a vector space and \(B, B' \subseteq V\) two finite
    bases of \(V\). Then \(|B| = |B'|\).

    D4.21:
    A vector space \(V\) is called finitely generated iff there exists a finite
    \(G \subseteq V\) with \(\Span(G) = V\).

    T4.22:
    Let \(V\) be a finitely generated vector space and let \(G \subseteq V\) be
    a finite subset with \(\Span(G) = V\). Then \(V\) has a basis
    \(B \subseteq G\).

    D4.23 (Dimension):
    Let \(V\) be a finitely generated vector space. Then \(\dim(V)\) is the
    size of any basis \(B\) of \(V\).

    L4.24: Let \(V\) be a vector space with \(\dim(V) = d\).
    \begin{enumerate}
        \item Let \(F \subseteq V\) be a set of linearly independent vectors.
            Then \(F\) is a basis of \(V\).
        \item Let \(G \subseteq V\) be a set of \(d\) vectors with
            \(\Span(G) = V\).
            Then \(G\) is a basis of \(V\).
    \end{enumerate}

    \subsection{Computing the fundamental subspaces}
    
    \subsubsection{Computing a basis of \(C(A)\) and \(R(A)\)}

    T4.25/T4.28:
    Let \(A \in \mathbb R^{m \times n}\) and \(R_0\) in
    REF(\(j_1, \ldots, j_r\)) the result of Gauss-Jordan elimination on \(A\) [T3.21].
    Then \(A\) has independent columns \(j_1, \ldots, j_r\) and these form the
    basis for \(C(A)\).
    The first \(r\) rows of \(R_0\) form a basis of \(R(A)\).
    Hence \(\dim(C(A)) = \dim(R(A)) = r = \rank(A)\).

    \subsubsection{}

    L4.27:
    Let \(A \in R^{m \times n}\) and \(M \in \mathbb R^{m \times m}\) invertible.
    Then \(R(A) = R(MA)\).

    T4.29:
    Let \(A \in \mathbb R^{m \times n}\). Then \(\rank(A) = \rank(A^\top)\).

    C4.30:
    Let \(A = CR\) be the CR decomposition \(A\) [T2.23].
    The columns of \(C\) form a basis of \(C(A)\) [T4.25].
    The rows of \(R\) form a basis of \(R(A)\) [T4.28], [T3.24].

    D4.31:
    Let \(A \in \mathbb R{m \times n}\).
    Then \(N(A) = \{x \in \mathbb R^n : Ax = 0\} \subseteq \mathbb R^n\)

    L4.32:
    Let \(A \in \mathbb R^{m \times n}\).
    Then \(N(A)\) is a subspace of \(\mathbb R^n\).

    L4.33:
    Let \(A \in \mathbb R^{m \times n}\) and \(M \in \mathbb R^{m \times m}\)
    invertible. Then \(N(A) = N(MA)\).

    \subsubsection{Computing a basis of \(N(A)\)}

    L4.34:
    Let \(R \in \mathbb R^{r \times n}\) be in \(\RREF(j_1, \ldots, j_r)\)
    (see [D3.19]). Let \(j_{r+1} < \cdots < j_n\) denote the indices of the
    dependent columns.
    The \(r \times r\) submatrix of \(R\) formed by the independent columns
    is \(I\). We let \(Q \in \mathbb R^{r \times (n-r)}\) denote the submatrix
    of \(R\) formed by the dependent columns.
    For \(x \in \mathbb R^n\), let
    
    \[
        x(I) = \begin{bmatrix}
            x_{j_1} \\
            \vdots \\
            x_{j_r}
        \end{bmatrix} \in \mathbb R^r
        \text{ and }
        x(Q) = \begin{bmatrix}
            x_{j_{r+1}} \\
            \vdots \\
            x_{j_{n}}
        \end{bmatrix} \in \mathbb R^{n-r} .
    \]

    denote the subvectors of basic and free entries.
    Let \(v_1, \ldots, v_{n-r} \in \mathbb R^n\) be the vectors
    defined via \(v_i(Q) = e_i\) and \(v_i(I) = -Qv_i(Q)\).
    Then \(\{v_1, \ldots, v_{n - r}\}\) is a basis of \(N(R)\).

    \subsubsection{}

    T4.35:
    Let \(A \in \mathbb R^{m \times n}\) and let \(R_0\) in
    \(\REF(j_1, \ldots, j_r)\) be the result of Gauss-Jordan elimination on \(A\)
    (see [T3.21]). Let \(R\) in \(\RREF(j_1, \ldots, j_r)\) be the submatrix of
    \(R_0\) consisting of the first \(r\) rows. The vectors
    \(v_1, \ldots, v_{n-r}\) as constructed in [L4.34] form a basis of
    \(N(A) = N(R_0) = N(R)\) and therefore \(\dim(N(A)) = n - r = n - \rank(A)\).

    L4.36:
    Let \(A \in \mathbb R^{m \times n}\).
    Then \(LN(A) := N(A^\top) \subseteq \mathbb R^m\).

    L4.32 (L4.37):
    Let \(A \in \mathbb R^{m \times n}\).
    Then \(N(A)\) (\(LN(A)\)) is a subspace of \(R^n\) (\(\mathbb R^m\)).

    \subsubsection{Computing a basis of \(LN(A)\)}

    T4.38:
    Let \(A \in \mathbb R^{m \times n}\) and let \(R_0 = MA\) in
    \(\REF(j_1, \ldots, j_r)\) be the result of Gauss-Jordan elimination on \(A\)
    (see [T3.21]). Then the last \(m - r\) rows \(w_{r+1}, \ldots, w_m\) of
    \(M \in \mathbb R^{m \times m}\) form a basis of \(LN(A)\), and therefore
    \(\dim(LN(A)) = m - r = m - \rank(A)\).

    \subsubsection{Solution space of \(Ax = b\)}

    D4.39 (Solution space):
    Let \(A \in \mathbb R^{m \times n}\), \(b \in \mathbb R^m\).
    Then \(\Sol(A, b) := \{x \in \mathbb R^n : Ax = b\} \subseteq \mathbb R^n\).

    T4.40:
    Let \(A \in \mathbb R^{m \times n}\), \(b \in \mathbb R^m\) and \(s\) a
    solution to \(Ax = b\). Then \(\Sol(A, b) = \{s + x : x \in N(A)\}\).

    L4.41:
    Let \(A \in \mathbb R^{m \times n}, \rank(A) = m\).
    Then \(Ax = b\) has a solution for every \(b \in \mathbb R^m\).

    \section{Orthogonality}

    \subsection{Orthogonality}
    T5.1.6
    Let \(A \in \mathbb R^{m \times n}\). \(N(A) = C(A^\top)^\bot = R(A)^\bot\).

    T5.1.7
    Let \(V, W\) be orthogonal subspaces of \(\mathbb R^n\).
    The following Statements are equivalent:
    \begin{enumerate}
        \item \(W = V^\bot\).
        \item \(\dim(V) + \dim(W) = n\).
        \item Every \(u \in \mathbb R^n\) can be written as \(u = v + w\)
            with unique \(v \in V, w \in W\).
    \end{enumerate}

    L5.1.8
    Let \(V\) be a subspace of \(\mathbb R^n\).
    Then \(V = (V^\bot)^\bot\).

    T5.1.1
    \(\{x \in \mathbb R^n \mid Ax = b\} = x_1 + N(A)\) where \(x_1 \in R(A)\)
    such that \(Ax_1 = b\).

    L5.1.11
    Let \(A \in \mathbb R^{m \times n}\). Then \(N(A) = N(A^\top A)\) and
    \(C(A^\top) = C(A^\top A)\).

    \subsection{Projections}

    L5.2.2:
    Let \(a \in \mathbb R^m_*\) and
    \(S = \{\lambda a \mid \lambda \in \mathbb R\} = C(a)\).
    \(\proj_S(b) = \frac{aa^\top}{a^\top a}b\).

    L5.2.3:
    Let \(b \in \mathbb R^m\) and \(S = C(A)\).
    Then \(\proj_S(b) = A\hat x\) where \(\hat x\) satisfies
    \(A^\top A \hat x = A^\top b\).

    L5.2.4:
    \(A^\top A\) is invertible iff \(A\) has linearly independent columns.

    T5.2.6: Let \(S\) be a subspace in \(\mathbb R^n\) and the columns of \(A\)
    are a basis of \(S\).
    Then \(\proj_S(b) = Pb\) where \(P = A(A^\top A)^{-1} A^\top\).

    R5.2.7:
    \begin{enumerate}
        \item \(P^2 = P\).
        \item \((I - P)b = \proj_{S^\bot}(b)\).
        \item \((I - P)^2 = I - P\).
    \end{enumerate}

    \subsection{Least squares approximation}

    (2):
    \(\min_{\hat x \in \mathbb R^n}\|A \hat x - b\|^2\).

    (3):
    \(A \top A \hat x = A^\top b\).

    F5.3.1:
    A minimizer of (2) is also a solution of (3).
    When \(A\) has independent columns the unique solution \(\hat x\)
    of (2) is given by \(\hat x = (A^\top A)^{-1} A^\top b\).

    \subsubsection{Linear regression}
    Problem: Consider data points \((t_1, b_1), \ldots, (t_m, b_m)\).
    Find \(\alpha_0, \alpha_1 \in \mathbb R\) such that
    \(b_k \approx \alpha_0 + \alpha_1 t_k\).

    Solution: Let

    \[
    b = \begin{bmatrix}
        b_1 \\ \vdots \\ b_m
    \end{bmatrix}
    , \ 
    A = \begin{bmatrix}
        1 & t_1 \\ \vdots & \vdots \\ 1 & t_m
    \end{bmatrix}
    \ \text{and} \ 
    \alpha =
    \begin{bmatrix}
        \alpha_0 \\ \alpha_1
    \end{bmatrix} \ .
    \]

    Then \(\alpha = (A^\top A)^{-1} A^\top b\) which can be written as

    \[
    \alpha = \begin{bmatrix}
        m & \sum_{k = 1}^m t_k \\
        \sum_{k = 1}^m t_k & \sum_{k = 1}^m t_k^2
    \end{bmatrix}^{-1}
    \begin{bmatrix}
        \sum_{k = 1}^m b_k \\
        \sum_{k = 1}^m t_k b_k
    \end{bmatrix} \ .
    \]

    % L5.3.2:
    % [Trivial].

    R5.3.3:
    If the columns of \(A\) are pairwise orthogonal,
    then \(A^\top A\) is a diagonal matrix, which is easy to invert.
    In this case this corresponds to \(\sum_{k = 1}^m t_k = 0\).
    Then the formula for \(\alpha\) simplifies to

    \[
    \alpha = \begin{bmatrix}
        \frac{1}{m} \sum_{k=1}^m b_k \\
        (\sum_{k=1}^m t_k b_k) / (\sum_{k=1}^m t_k^2)
    \end{bmatrix} \ .
    \]

    \subsection{Orthonormal bases, Gram-Schmidt}

    D5.4.1:
    Vectors \(q_1, \ldots, q_n \in \mathbb R^m\) are \textit{orthonormal} iff
    they are orthogonal and have norm 1.
    In other words, for all \(i, j \in [n]\) we have
    \(q_i^\top q_j = \delta_{ij}\).
    In this case for \(Q\) with columns \(q_i\) we have \(Q^\top Q = I\).
    
    D5.4.3:
    \(Q \in \mathbb R^{n \times n}\) is orthogonal iff \(Q^\top Q = I\).
    In this case  \(QQ^\top = I, Q^{-1} = Q^\top\) and the columns form
    an orthonormal basis for \(\mathbb R^n\).

    P5.4.6:
    Let \(Q \in \mathbb R^{m \times m}\) be orthogonal and
    \(x, y \in \mathbb R^m\). Then \(\|Qx\| = \|x\|\) and
    \((Qx)^\top (Qy) = x^\top y\).

    P5.4.7:
    Let \(S\) be a subspace of \(\mathbb R^m\) and \(q_1, \ldots, q_n\)
    be an orthonormal basis for \(S\).
    Let \(Q \in \mathbb R^{m \times n}\) with columns \(q_i\).
    Then the projection matrix that projects onto \(S\) is \(QQ^\top\)
    and the Least squares solution to \(Qx = b\) is \(\hat x = Q^\top b\).

    \subsubsection{Gram-Schmidt process}

    A5.4.9:
    Let \(a_1, \ldots, a_n\) be linearly independent.
    Then
    \begin{itemize}
        \item \(q_1 = \frac{a_1}{\|a_1\|}\).
        \item For \(k = 2, \ldots, n\) set \\
            \(q_k' = a_k - \sum_{i=1}^{k-1}(a_k^\top q_i) q_i\) \\
            \(q_k = \frac{q_k'}{\|q_k'\|}\).
    \end{itemize}

    T5.4.10:
    The Gram-Schmidt process returns an orthonormal basis for the span of
    \(a_1, \ldots, a_n\).

    \subsubsection{QR decomposition}

    D5.4.11:
    Let \(A \in \mathbb R^{m \times n}\) have linearly independent columns.
    The \textit{QR decomposition} is \(A = QR\) where \(Q \in \mathbb R^{m \times n}\)
    is orthonormal (the output of [A5.4.9]) and \(R = Q^\top A\).

    L5.4.12:
    In [D5.4.11] \(R\) is upper triangular und invertible.
    Moreover \(QQ^\top A = A\) and hence \(A = QR\) is well defined.

    F5.4.13:
    The QR decomposition simplifies some calculations:
    \begin{itemize}
        \item \(C(A) = C(Q)\) leads to \(\proj_{C(A)}(b) = QQ^\top b\).
        \item \(A^\top A \hat x = A^\top b\) becomes
            \(R \hat x = Q^\top b\).
    \end{itemize}

    \subsection{Pseudoinverse}

    D5.5.1:
    Let \(A \in \mathbb R^{m \times n}, \rank(A) = n\).
    Then the \textit{pseudoinverse} \(A^+ \in \mathbb R^{n \times m}\)
    of \(A\) is \(A^+ = (A^\top A)^{-1} A^\top\).

    P5.5.2:
    Let \(A \in \mathbb R^{m \times n}, \rank(A) = n\).
    Then \(A^+ A = I\).

    D5.5.3:
    Let \(A \in \mathbb R^{m \times n}, \rank(A) = m\).
    Then \(A^+ = A^\top (AA^\top)^{-1}\).

    L5.5.4:
    Let \(A \in \mathbb R^{m \times n}, \rank(A) = m\).
    Then \(A A^+ = I\).

    (10):
    \(\min_{x \in \mathbb R^n}\|x\|^2\).

    L5.5.5:
    Let \(A \in \mathbb R^{n \times n}\), \(b \in C(A)\),
    the (unique) solution to (10)
    is given by \(\hat x \in C(A^\top)\) that satisfies the
    constraint \(A \hat x = b\).

    P5.5.6:
    For a full row rank matrix \(A\), the unique solution to (10)
    is given by \(\hat x = A^+ b\).

    D5.5.7:
    Let \(A \in \mathbb R^{m \times n}\) with CR decomposition \(A = CR\).
    Then \(A^+ = R^+ C^+ = R^\top (C^\top A R^\top)^{-1} C^\top\).

    L5.5.8:
    Let \(A \in \mathbb R^{m \times n}, b \in \mathbb R^n\).
    The (unique) solution to \(\min_{x \in \mathbb R^n} \|x\|^2\) s.t.
    \(A^\top Ax = A^\top b\), is given by \(\hat x = A^+ b\).

    P5.5.9:
    Let \(A \in \mathbb R^{m \times n}\), \(\rank(A) = r\) and
    \(S \in \mathbb R^{m \times r}\), \(T \in \mathbb R^{r \times n}\)
    such that \(A = ST\). Then \(A^+ = T^+ S^+\).
    
    T5.5.11:
    Let \(A \in \mathbb R^{m \times n}\).
    \begin{enumerate}
        \item \(AA^+A = A\).
        \item \(A^+AA^+ = A^+\).
        \item \(AA^+\) is symmetric and projects on \(C(A)\).
        \item \(A^+A\) is symmetric and projects on \(C(A^\top)\).
        \item \((A^\top)^+ = (A^+)^\top\)
    \end{enumerate}

    P5.5.12:
    Let \(A \in \mathbb R^{m \times n}\).
    Then \(f : C(A^\top) \to C(A)\), \(f : x \mapsto Ax\) is a bijection.

    \subsection{Farkas' lemma}
    
    D5.6.1:
    Let \(A \in \mathbb Q^{m \times n}\), \(b \in \mathbb Q^m\) and
    \(P = \{x \in \mathbb R^n \mid Ax \le b\}\). \(P\) is called a
    \textit{polyhedron}. Let \(S = [s]\). The \textit{projection of \(P\)}
    on the subspace \(\mathbb R^s\) associated with the variables in the subset
    \(S\) is \(\proj_S(P) := \{x \in \mathbb R^s \mid \exists y \in \mathbb R^{n-s}
    \text{ such that } (x, y) \in P\}\).

    P5.6.2:
    \(P \neq \emptyset \iff l \le u \iff 0 \le u - l \iff 0 \le y^\top b\)
    for all \(y \ge 0\) such that \(y^\top a = 0\).

    A:
    Let \(A \in \mathbb Q^{m \times n}\), \(b \in \mathbb Q^m\) and
    \(P = \{x \in \mathbb R^n \mid Ax \le b\}\).
    Let the entries of \(A\) be denoted by \(a_{ij}\).
    Then row \(i\) gives us the inequality \(\sum_{j=1}^n a_{ij} x_j \le b_i\).
    Let \(\bar x = (x_1, \ldots, x_{n-1})\) and \(\bar A\) consist of the first
    \(n - 1\) columns of \(A\). Consider the following algorithm.
    \begin{enumerate}
        \item Partition the indices \(M = [m]\) of the rows into three subsets
            \(M_0 = \{i \in M \mid a_{i,n} = 0\}\), 
            \(M_+ = \{i \in M \mid a_{i,n} > 0\}\) and
            \(M_- = \{i \in M \mid a_{i,n} < 0\}\). 
        \item
            \begin{itemize}
                \item For every row with index \(i \in M_+\) multiply the
                    corresponding constraint by \(\frac{1}{a_{in}}\).
                    This gives a new representation of row \(i\) as
                    \(x_n \le d_i + f_i^\top \bar x\) for \(i \in M_+\) where
                    \(d_i = \frac{b_i}{a_{in}}\), \(f_{ij} = - \frac{a_{ij}}{a_{in}}\).
                \item Every row with index \(k \in M_0\) can be rewritten as
                    \(0 \le d_k + f_k^\top \bar x \text{ for } k \in M_0\) where
                    \(d_k = b_k\), \(f_{kj} = -a_{kj}\).
                \item For every row with index \(i \in M_-\) multiply the
                    corresponding constraint by \(\frac{1}{a_{in}}\).
                    This gives a new representation of row \(i\) as
                    \(x_n \ge d_i + f_i^\top \bar x\) for \(i \in M_-\) where
                    \(d_i = \frac{b_i}{a_{in}}\), \(f_{ij} = - \frac{a_{ij}}{a_{in}}\).
            \end{itemize}

        \item Return \(Q = \{\bar x \in \mathbb R^{n-1} \mid 
            0 \le d_k + f_k^\top \bar x \text{ for all } k \in M_0,
            d_l + f_k^\top \bar x \le d_i + f_i^\top \bar x \text{ for all }
            l \in M_-, i \in M_+\}\)
    \end{enumerate}

    T5.6.3:
    The set \(Q\) returned in Step 3 is a polyhedron.
    Moreover \(Q = \proj_S(P)\), where \(S = [n-1]\).

    L5.6.4:
    Let \(A \in \mathbb Q^{m \times n}\), \(b \in \mathbb Q^n\) and
    \(P = \{x \in \mathbb R^n \mid Ax \le b\}\).
    Let \(S_1 = [n-1]\) and \(S_2 = [n-2]\).
    Then \(\proj_{S_2}(P) = \proj_{S_2}(\proj_{S_1}(P))\).

    D5.6.5:
    Let \(A \in \mathbb Q^{m \times n}\), \(b \in \mathbb Q^m\) and
    \(P = \{x \in \mathbb R^n \mid Ax \le b\}\).
    For \(k \in [j]\) let \(A^{(j)}\) be the submatrix of \(A\) with
    column vectors \(A._k\).
    Let \(P^{(0)} = P\) and \(C^{(0)} = \mathbb R^m_+\).
    Define for \(i \in [n]\) = \(C^{(i)} = (1)\) and \(P^{(i)} = (2)\)

    \begin{align*}
        (1) &= \{y \in \mathbb R^m_+ \mid y^\top A._k = 0 \text{ for all }
        k = n - i + 1, \ldots, n\} \\
            (2) &= \{\hat x \in \mathbb R^{n-1} \mid y^\top A^{(n-i)} \hat x
        \le y^\top b \text{ for all } y \in C^{(i)}\}
    \end{align*}

    T5.6.6:
    \(\proj_{S_{n-i}}(P) = P^{(i)}\).

    \subsubsection{Farkas' lemma}

    T5.6.7:
    Let \(A \in \mathbb Q^{m \times n}\), \(b \in \mathbb Q^m\).
    Either there exists an \(x \in \mathbb R^n\) such that \(Ax \le b\)
    or there exists a \(y \in \mathbb R^m\) such that \(y \ge \mathbf 0\),
    \(y^\top A = \mathbf 0\) and \(y^\top b < 0\).

    \section{Determinant}

    D6.0.4:
    Let \(\sigma : [n] \to [n]\) be a permutation of \(n\) elements.
    The sign \(\sgn(\sigma)\) counts the parity of the numer of pairs of elements
    that are out of order after applying \(\sigma\) (1 if even, -1 if odd).

    D6.0.6:
    Let \(A \in \mathbb R^{n \times n}\).
    The \textit{determinant} is defined as
    \(\det(A) := \sum_{\sigma \in \Pi_n} \sgn(\sigma)
    \prod_{i=1}^n A_{i, \sigma(i)}\), where \(\Pi_n\)
    is the set of all permutations of \(n\) elements.

    P6.0.7:
    Let \(P \in \mathbb R^{m \times m}\) be the permutation matrix
    corresponding to \(\sigma\).
    Then \(\det(P) = \sgn(\sigma)\).

    P6.0.8:
    Let \(T \in \mathbb R^{m \times m}\) be triangular.
    Then \(\det(T) = \prod_{k=1}^n T_{kk}\).
    In particular \(\det(I) = 1\).

    T6.0.9:
    Let \(A \in \mathbb R^{m \times m}\).
    Then \(\det(A^\top) = \det(A)\).

    P6.0.10:
    Let \(Q \in \mathbb R^{m \times m}\) be orthogonal.
    Then \(\det(Q) = \pm 1\).

    P6.0.11:
    A matrix \(A \in \mathbb R^{m \times m}\) is invertible iff \(\det(A) \neq 0\).

    P6.0.12:
    Let \(A, B \in \mathbb R^{m \times m}\).
    Then \(\det(AB) = \det(A) \det(B)\).

    P6.0.13:
    Let \(A, B \in \mathbb R^{m \times m}\) with \(\det(A) \neq 0\).
    Then \(\det(A^{-1}) = \frac{1}{\det(A)}\).

    D5.0.15:
    Let \(A \in \mathbb R^{m \times m}\) and let \(\mathscr A_{ij}\) denote
    the \((m - 1) \times (m - 1)\) matrix obtained by removing row \(i\)
    and column \(j\) from \(A\).
    Then we define the co-factors of \(A\) as
    \(C_{ij} = (-1)^{i+j} \det(\mathscr A_{ij})\).

    P5.0.16:
    Let \(A \in \mathbb R^{m \times m}\), \(i \in [n]\). Then
    \(\det(A) = \sum_{j=1}^n A_{ij} C_{ij}\).

    P5.0.17:
    Let \(A \in \mathbb R^{m \times m}\) with \(\det(A) \neq 0\) and
    \(C\) the matrix with the cofactors of \(A\).
    Then \(A^{-1} = \frac{1}{\det(A)}C^\top\).

    P6.0.19 (Cramer's rule):
    Let \(A \in \mathbb R^{n \times n}\), \(b \in \mathbb R^n\) with
    \(\det(A) \neq 0\). Then the solution to \(Ax = b\) is given by
    \(x_j = \frac{\det(\mathscr B_j)}{\det(A)}\) where \(\mathscr B_j\)
    is obtained by replacing the \(j\)th column of \(A\) with \(b\).

    P6.0.21:
    Let \(A \in \mathbb R^{n \times n}\) and \(P\) a permutation
    that swaps two elements. Then \(\det(PA) = - \det(A)\).

    P6.0.22:
    The determinant is linear in each row or each column.
    In other words, for any \(a_0, \ldots, a_n \in \mathbb R^n\) and
    \(\alpha_0, \alpha_1 \in \mathbb R\) we have
    
    \begin{align*}
        &\begin{vmatrix}
            \text{---} & \alpha_0 a_0^\top + \alpha_1 a_1^\top & \text{---} \\
            \text{---} & a_2^\top & \text{---} \\
                   & \vdots &  \\
            \text{---} & a_n^\top & \text{---} \\
        \end{vmatrix}
        = \\
        &\alpha_0
        \begin{vmatrix}
            \text{---} & a_0^\top & \text{---} \\
            \text{---} & a_2^\top & \text{---} \\
                   & \vdots &  \\
            \text{---} & a_n^\top & \text{---} \\
        \end{vmatrix}
        +
        \alpha_1
        \begin{vmatrix}
            \text{---} & a_1^\top & \text{---} \\
            \text{---} & a_2^\top & \text{---} \\
                   & \vdots &  \\
            \text{---} & a_n^\top & \text{---} \\
        \end{vmatrix}
    \end{align*}
    
    and the same but transposed. 

    \section{Eigenvalues and Eigenvectors}

    \subsection{Complex numbers}

    D (Complex numbers):
    \begin{itemize}
        \item \((a + ib) + (x + iy) =  (a + x) + i(b + y)\)
        \item \((a + ib)(x + iy) = (ax - by) + i(ay + bx)\)
        \item \((a + ib)(a - ib) = a^2 + b^2\)
        \item \(\frac{a + ib}{x + iy}
            = \frac{(x - iy)(a + ib)}{(x - iy)(x + iy)}
            = \frac{ax + by}{x^2 + y^2} + i \frac{bx - ay}{x^2 + y^2}\)
    \end{itemize}


    D (Notation):
    \begin{itemize}
        \item \(\mathfrak R(a + ib) := a\)
        \item \(\mathfrak I(a + ib) := b\)
        \item \(|z| := \sqrt{a^2 + b^2}\) (\textit{modulus})
        \item \(\overline{a + ib} := a - ib\) (\textit{complex conjugate})
    \end{itemize}

    F7.0.1:
    Let \(\theta \in \mathbb R\).
    Then \(e^{i \theta} = \cos \theta + i \sin \theta\).

    F7.0.2:
    A complex number \(z \in \mathbb C\) can be written as
    \(z = re^{i \theta}\) where \(r \ge 0\) is the \textit{modulus}
    and \(\theta \in \mathbb R\) is the \textit{argument}.

    T7.0.3 (Fundamental theorem of algebra):
    Any degree \(n\) non-constant polynomial
    \(P(z) = \alpha_n z^n + \cdots + \alpha_1 z + \alpha_0\)
    with \(\alpha_n \neq 0\) has a zero: \(\lambda \in \mathbb C\)
    such that \(P(\lambda) = 0\).

    C7.0.4:
    Any degree \(n\) non-constant polynomial
    \(P(z) = \alpha_n z^n + \cdots + \alpha_1 z + \alpha_0\) has \(n\) zeros:
    \(\lambda_1, \ldots, \lambda_n \in \mathbb C\), perhaps with repetitions,
    such that \(P(z) = \alpha_n (z - \lambda_1) \cdots (z - \lambda_n)\).
    The number of times \(\lambda \in \mathbb C\) appears in the expansion
    is called the \textit{algebraic multiplicity} of the zero.

    D:
    Let \(A \in \mathbb C^{m \times n}\). Then \(A^* := \overline A^\top\).

    F:
    Let \(v, w \in \mathbb C^n\).
    Then \(\|v\|^2 = v^* v = \bar v\top v\).
    Furthermore \(\langle v, w \rangle = w^* v\).

    \subsection{Introduction to Eigenvalues and Eigenvectors}

    Problem:
    Find the explicit representation of the linear recurrence
    \(F_0 = 0, F_1 = 1\) and \(F_n = F_{n-1} + F_{n-2}\).

    Solution:
    Let

    \[
        M =
        \begin{bmatrix}
            1 & 1 \\
            1 & 0
        \end{bmatrix}
        \text{, }
        g_0 =
        \begin{bmatrix}
            1 \\
            0
        \end{bmatrix}
        \text{ and }
        g_n =
        \begin{bmatrix}
            F_n \\
            F_{n-1}
        \end{bmatrix} .
    \] \\

    Then \(g_n = M g_{n-1} = \cdots = M^n g_0\).
    We now solve \(0 = \det(M - \lambda I)\) and get
    \(\lambda_1 = \frac{1 + \sqrt 5}{2}\) and
    \(\lambda_2 = \frac{1 - \sqrt 5}{2}\).
    We find \(v_k\), the non-zero element of \(N(M - \lambda_k I)\)
    for \(k = 1, 2\). We write \(g_0 = \alpha_1 v_1 + \alpha_2 v_2\)
    and get \(\alpha_1 = \frac{1}{\sqrt 5}\) and \(\alpha_2 = -\frac{1}{\sqrt 5}\).
    Then \(g_n = A^n g_0 = A^n (\alpha_1 v_1 + \alpha_2 v_2) =
    \alpha_1 \lambda_1^n v_1 + \alpha_2 \lambda_2^n v_2\).

    D7.1.1:
    Let \(A \in \mathbb R^{n \times n}\).
    Then \(\lambda \in \mathbb C\) is an \textit{eigenvalue (EW)} and
    \(v \in \mathbb C^n \setminus \{0\}\) is an \textit{eigenvector (EV)} of \(A\)
    iff \(A v = \lambda v\).

    P7.1.2:
    Let \(A \in \mathbb R^{n \times n}\).
    Then \(\lambda \in \mathbb R\) is a (real) EW of \(A\)
    iff \(\det(A - \lambda I) = 0\) and
    \(v \in \mathbb R^n\) is an EV associated with \(\lambda\) iff it is
    a nonzero element of \(N(A - \lambda I)\).

    P7.1.3:
    \(\det(A - \lambda I)\) is a polynomial in \(\lambda\) of degree \(n\).
    The coefficient of the \(\lambda^n\) term is \((-1)^n\).

    T7.1.4:
    Every matrix \(A \in \mathbb R^{n \times n}\) has an EW
    (perhaps in \(\mathbb C\)).

    P7.1.6:
    Let \(\lambda\) and \(v\) be an EW-EV pair of the matrix \(A\).
    Then for \(k \ge 1\), \(\lambda^k\) and \(v\) are an EW-EV pair
    of the matrix \(A^k\).

    P7.1.7:
    Let \(\lambda\) and \(v\) be an EW-EV pair of the
    invertible matrix \(A\).
    Then \(\frac{1}{\lambda}\) and \(v\) are an EW-EV pair of the
    matrix \(A^{-1}\).

    P7.1.8:
    Let \(A \in \mathbb R^{n \times n}\) and let
    \(v_1, \ldots, v_k \in \mathbb R^n\) be EWs corresponding
    to EVs \(\lambda_1, \ldots, \lambda_k \in \mathbb R\).
    If \(\lambda_1, \ldots, \lambda_k \in \mathbb R\) are pairwise distinct,
    the EVs \(v_1, \ldots, v_k\) are linearly independent.

    T7.1.9:
    Let \(A \in \mathbb R^{n \times n}\) with \(n\) pairwise distinct
    real EVs (see [T7.1.8], [C7.0.4]) then there is a basis of
    \(\mathbb R^n\) made up of the EVs of \(A\).

    P7.1.10:
    Let \(A \in \mathbb R^{n \times n}\).
    Then \(A\) and \(A^\top\) have the same EWs.

    D7.1.11:
    Let \(A \in \mathbb R^{n \times n}\).
    The \textit{trace} of \(A\) is \(\Tr(A) := \sum_{i = 1}^n A_{ii}\).

    (33) (Characteristic polynomial of \(A\)):
    \((-1)^n \det(A - zI) = \det(zI - A) =
        (z - \lambda_1) \cdots (z - \lambda_n)\)

    P7.1.12
    Let \(A \in \mathbb R^{n \times n}\) and \(\lambda_1, \ldots, \lambda_n\)
    its \(n\) EWs as they show up in (33) (meaning that a value may be repeated).
    Then \(\Tr(A) = \sum_{i=1}^n \lambda_i\)
    and \(\det(A) = \prod_{i=1}^n \lambda_i\).

    R7.1.13:
    [P7.1.12] can be useful to check computations.

    L7.1.14:
    Let \(A, B, C \in \mathbb R^{n \times n}\). Then
    \begin{enumerate}
        \item \(\Tr(AB) = \Tr(BA)\),
        \item \(\Tr(ABC) = \Tr(BCA) = \Tr(CAB)\).
    \end{enumerate}

    R7.1.15:
    Important words of caution:
    \begin{enumerate}
        \item The EWs of \(A\) and \(A^\top\) are the same, the EVs might not!
        \item The EWs of \(A + B\) are not easily computed from the EWs of
            \(A\) and \(B\)!
        \item The EWs of \(A \cdot B\) are not easily computed from the EWs of
            \(A\) and \(B\)!
        \item Gauss elimination does not preserve EWs or EVs!
            The EWs are not the diagonal elements of \(U\) in the \(PA = LU\)
            factorization.
    \end{enumerate}

    P7.1.17:
    Let \(Q \in \mathbb R^{n \times n}\) be orthogonal and
    \(\lambda \in \mathbb C\) an EW of \(Q\).
    Then \(|\lambda| = 1\).

    D7.1.20:
    If, given \(A \in \mathbb R^{n \times n}\) we can build a basis of
    \(\mathbb R^n\) with EVs of \(A\), we say that \(A\) has a complete
    set of real EVs.

    P7.1.21:
    Let \(P\) be a projection matrix. Then \(P\) has at most two EWs, 0 and 1,
    and a complete set of real EVs.

    D7.1.22:
    Let \(A \in \mathbb R^{n \times n}\) and \(\lambda\) an EW of \(A\).
    Then we call the dimension of \(N(A - \lambda I)\) the
    \textit{geometric multiplicity} of \(\lambda\).
    
    \subsection{Diagonalizing and change of basis}

    T7.2.1:
    Let \(A \in \mathbb R^{n \times n}\) be a matrix with a complete set of
    real EVs (see [D7.1.20]) and let \(v_1, \ldots, v_n \in \mathbb R^n\)
    be a basis formed with EVs of \(A\) and let \(\lambda_1, \ldots, \lambda_n\)
    be the associated EWs.
    Let \(V\) be the matrix whose columns are the \(v_i\)s.
    Then \(A = V \Lambda V^{-1}\), where \(\Lambda\) is a diagonal matrix with
    \(\Lambda_{ii} = \lambda_i\).

    D7.2.2:
    \(A \in \mathbb R^{n \times n}\) is called \textit{diagonalizable} iff
    there exists an invertible matrix \(V\) such that
    \(V^{-1} A V = \Lambda\), where \(\Lambda\) is a diagonal matrix.

    D7.2.3:
    \(A, B \in \mathbb R^{n \times n}\) are called \textit{similar} iff
    there exists an invertible matrix \(S\) such that \(B = S^{-1} A S\).

    P7.2.4:
    Similar matrices have the same EWs.

    R7.2.5:
    If we have a matrix \(A \in \mathbb R^{n \times n}\) with a complete set
    of real EVs then [T7.2.1] tells us that the corresponding linear transformation,
    when viewed in the basis \(v_1, \ldots, v_n\) is simply a diagonal matrix.

    \subsection{Symmetric matrices and Spectral theorem}

    \subsubsection{Spectral theorem}

    T7.3.1:
    Let \(A \in \mathbb R^{n \times n}\) be symmetric.
    Then \(A\) has \(n\) real EWs and an orthogonal basis made of
    EVs of \(A\).

    C7.3.2:
    Let \(A \in \mathbb R^{n \times n}\) be symmetric.
    Then there exists an orthogonal matrix \(V \in \mathbb R^{n \times n}\)
    (whose columns are EVs of \(A\)) such that \(A = V \Lambda V^\top\),
    where \(\Lambda \in \mathbb R^{n \times n}\) is a diagonal matrix with
    the eigenvalues of \(A\) on its diagonal.

    R7.3.3: The decomposition in [C7.3.2] and [T7.2.1] is called
    \textit{eigendecomposition}.

    C7.3.4:
    The rank of a real symmetric matrix \(A\) is the number of non-zero
    eigenvalues (counting repetitions).

    R7.3.5:
    Let \(A \in \mathbb R^{n \times n}\).
    Then \(\rank(A) = n - \dim(N(A))\) which is the geometric multiplicity of
    \(\lambda = 0\) (see [D7.1.22]).
    Since symmetric matrices always have a complete set of EWs and EVs,
    the geometric multiplicities are always the same as the algebraic
    multiplicities.

    P7.3.6:
    Let \(A \in \mathbb R^{n \times n}\) be symmetric and let
    \(v_1, \ldots, v_n\) be an orthonormal basis of EVs of \(A\)
    (the columns of \(V\) in [C7.3.2]) and \(\lambda_1, \ldots, \lambda_n\)
    the associated EWs. Then \(A = \sum_{k=1}^n \lambda_i v_i v_i^\top\).

    P7.3.7:
    Let \(A \in \mathbb R^{n \times n}\) be a symmetric matrix and
    \(\lambda \in \mathbb C\) an EW of \(A\).
    Then \(\lambda \in \mathbb R\).

    C7.3.8:
    Every symmetric matrix \(A \in \mathbb R^{n \times n}\) has a
    real EW \(\lambda\).

    \subsubsection{Rayleigh quotient}

    P7.3.10:
    Let \(A \in \mathbb R^{n \times n}\).
    Then the \textit{Rayleigh quotient} defined for
    \(x \in \mathbb R^n \setminus \{0\}\) as
    \(R(x) := \frac{x \top A x}{x \top x}\)
    attains its maximum at \(R(v_{\max}) = \lambda_{\max}\)
    and its minimum at \(R(v_{\min}) = \lambda_{\min}\)
    where \(\lambda_{\max}\) (\(\lambda_{\min}\)) is the
    largest (smallest) EW.

    \subsubsection{}

    D7.3.11:
    \(A \in \mathbb R^{n \times n}\) is said to be
    \textit{positive semidefinite (PSD)}
    (\textit{positive definite (PD)}) iff all its EWs are
    non-negative (strictly positive).

    P7.3.12:
    Let \(A \in \mathbb R^{n \times n}\) be symmetric.
    Then \(A\) is PSD (PD) iff \(x^\top A x \ge 0\) (\(x^\top A x > 0\))
    for all \(x \in \mathbb R^n \setminus \{0\}\).

    F7.3.13:
    Let \(A, B \in \mathbb R^{n \times n}\) be PSD (PD).
    Then their sum is PSD (PD).

    D7.3.14 (Gram matrix):
    Let \(v_1, \ldots, v_n \in \mathbb R^m\).
    We define their \textit{Gram matrix} \(G \in \mathbb R^{n \times n}\) as
    \(G_{ij} = v_i^\top v_j\).
    Note that if \(V \in \mathbb R^{m \times n}\) has the \(v_i\)s as columns
    then \(G = V^\top V\).
    
    R7.3.15:
    Let \(A \in \mathbb R^{m \times n}\).
    As an abuse of notation we also call \(AA^\top\) a Gram matrix of \(A\).
    If \(a_1, \ldots, a_n \in \mathbb R^m\) are the columns of \(A\) then
    \(AA^\top \in \mathbb R^{m \times m}\) and
    \(AA^\top = \sum_{i=1}^n a_i a_i^\top\).

    P7.3.16:
    Let \(A \in \mathbb R^{m \times n}\).
    Then the non-zero EWs of \(A^\top A \in \mathbb R^{n \times n}\)
    are the same as the ones of \(AA^\top \in \mathbb R^{m \times m}\).
    Both matrices are symmetric and PSD.

    \subsubsection{Cholesky decomposition}

    P7.3.17:
    Every symmetric PSD matrix \(M\) is a gram matrix of an upper triangular
    matrix \(C\). \(M = C^\top C\) is known as the \textit{Cholesky decomposition}.
    
    \section{Singular value decomposition}

    \subsection{Singular value decomposition (SVD)}

    D8.1.1:
    Let \(A \in \mathbb R^{m \times n}\).
    There exist orthogonal matrices \(U \in \mathbb R^{m \times m}\) and
    \(V \in \mathbb R^{n \times n}\) such that
    \(A = U \Sigma V^\top\), where \(\Sigma \in \mathbb R^{m \times n}\)
    is a diagonal matrix, where the diagonal elements are non-negative and
    ordered in descending order.
    The columns \(u_1, \ldots, u_m\) (\(v_1, \ldots, v_n\)) of
    \(U\) (\(V\)) are called the
    \textit{left} (\textit{right}) \textit{singular vectors} of \(A\) and are orthonormal.
    The diagonal elements of \(\Sigma\), \(\sigma_i = \Sigma_{ii}\)
    are called the \textit{singular values} of \(A\) and are ordered as
    \(\sigma_1 \ge \cdots \ge \sigma_{\min \{m, n\}}\).

    R8.1.2:
    Let \(A \in \mathbb R^{m \times n}\), \(\rank(A) = r\).
    We can write the SVD in a more compact form
    \(A = U_r \Sigma_r V_r^\top\) where \(U_r \in \mathbb R^{m \times r}\)
    (\(V_r \in \mathbb R^{n \times r}\)) contains the first \(r\)
    left (right) singular vectors, and \(\Sigma_r \in \mathbb R^{r \times r}\)
    is a diagonal matrix with the first \(r\) singular values.
    This requires considerably less space for a large matrix with small rank.

    R8.1.3:
    Let \(A \in \mathbb R^{m \times n}\) and \(A = U \Sigma V^\top\)
    be its SVD [D8.1.1]. Then \(AA^\top = U (\Sigma \Sigma^\top) U^\top\).
    Thus the left singular vectors of \(A\), the columns of \(U\) are the EVs
    of \(AA^\top\) and the singular values of \(A\) are the square-roots of the EWs
    of \(AA^\top\). Note that \(\Sigma \Sigma ^\top \in \mathbb R^{m \times m}\)
    is diagonal. If \(m > n\), \(A\) has \(n\) singular values and \(AA^\top\) has
    \(m\) EWs (which are larger than \(n\) but the "missing" ones are 0).
    Analogously, \(A^\top A = V(\Sigma^\top \Sigma) V^\top)\),
    and so the right singular vectors of \(A\), the columns of \(V\), are the EVs
    of \(A^\top A\) and the singular values of \(A\) are the square-roots of the EWs
    of \(A^\top A\). Note that \(\Sigma^\top \Sigma \in \mathbb R^{n \times n}\)
    is diagonal. If \(n > m\) \(A\) has \(m\) singular values and \(A^\top A\) has
    \(n\) EWs (which are larger than \(m\) but the "missing" ones are 0).
    This observation makes it easier to write the singular values/vectors of \(A\)
    in terms of EWs and EVs of \(AA^\top\) and \(A^\top A\), which are symmetric.
    This directly implies the uniqueness of singular values and the fact that the
    rank of a matrix is the number of non-zero singular values.

    P8.1.4:
    Let \(A \in \mathbb R^{m \times n}\), \(\rank(A) = r\).
    Let \(\sigma_1, \ldots, \sigma_r\) be the non-zero singular values of
    \(A\), \(u_1, \ldots, u_r\) (\(v_1, \ldots, v_r\)) the corresponding
    left (right) singular vectors.
    Then \(A = \sum_{k=1}^r \sigma_k u_k v_k^\top\).

    T8.1.5:
    Every \(A \in \mathbb R^{m \times n}\) has an SVD
    (see [D8.1.1]).

    \subsection{Vector and matrix norms}

    D (\(l_p\)-norm):
    For \(1 \le p \le \infty\) the \(l_p\)-norm is given by
    \(\|x\|_p := \left(\sum_{i=1}^n |x_i|^p\right)^{1/p}\).

    D8.2.1 (Frobenius and Spectral norm):
    Let \(A \in \mathbb R^{m \times n}\). Then:
    \begin{enumerate}
        \item \(\|A\|_F = \sqrt{\sum_{i=1}^m \sum_{j=1}^n A_{ij}^2}\),
        \item \(\|A\|_{op} = \max_{x \in \mathbb R^n \text{ s.t. } \|x\| = 1n}
            \|Ax\|\).
    \end{enumerate}

    P:
    Let \(A \in \mathbb R^{m \times n}\) with singular values
    \(\sigma_1, \ge \cdots \ge \sigma_{\min \{m, n\}}\). Then:
    \begin{enumerate}
        \item \(\|A\|^2_F = \Tr(A^\top A)\)
        \item \(\|A\|^2_F = \sum_{i=1}^{\min \{m, n\}} \sigma_i^2\)
        \item \(\|A\|_{op} = \sigma_1\)
        \item \(\|A\|_{op} \le \|A\|_F \le \sqrt{\min\{m,n\}} \|A\|_{op}\).
    \end{enumerate}

    \vspace{5em}

    \section{Appendix}

    \subsection{Notation}

    R:
    The word \textit{iff} stands for "if and only if".

    R:
    The abbreviation \textit{s.t.} stands for "such that".

    D:
    Let \(n \in \mathbb Z^+\). Then \([n] := \{1, \ldots, n\}\).

    D:
    A function is \textit{bijective} iff it is invertible.

    D:
    Let \(V\) be a vector space.
    Then \(V^\bot\) denotes the orthogonal complement of \(V\).

    D (Kronecker delta):

    \[
        \delta_{ij} := \begin{cases}
            1 & \text{if } i = j \\
            0 & \text{otherwise}
        \end{cases}
    \]

    D:
    Let \(A \in \mathbb R^{m \times n}\) with entries \(a_{ij}\). Then

    \[
        \begin{vmatrix}
            a_{1,1} & \cdots & a_{1,n} \\
            \vdots  & \ddots & \vdots  \\
            a_{m,1} & \cdots & a_{m,n} \\
        \end{vmatrix} = \det(A) .
    \]

    \subsection{Symbols}

    \begin{center}
        \begin{tabular}{|l|l|}
            \hline
            A & Algorithm \\
            \hline
            C & Corollary \\
            \hline
            D & Definition \\
            \hline
            F & Fact \\
            \hline
            L & Lemma \\
            \hline
            O & Observation \\
            \hline
            P & Proposition \\
            \hline
            R & Remark \\
            \hline
            T & Theorem \\
            \hline
        \end{tabular}
    \end{center}

    % further layouts

    % two-column layout within the section
    % \begin{minipage}{0.49\linewidth}
    %     Left
    % \end{minipage}
    % \hfill
    % \begin{minipage}{0.49\linewidth}
    %    Right
    % \end{minipage}

    % \section{Boxed Equation}
    % % Boxed equation with a simple math expression
    % \mathbox{
    %     $E = mc^2$
    % }

    % \section{Input}
    % Including an external file
    % \input{sections/00_Example}

    % \section{Image}
    % Including an image with full width
    % \includegraphics[width=\linewidth]{img/example.png} 
}
